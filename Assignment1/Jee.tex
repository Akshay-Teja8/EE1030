\let\negmedspace\undefined
\let\negthickspace\undefined
\documentclass[journal,12pt,twocolumn]{IEEEtran}
\usepackage{cite}
\usepackage{amsmath,amssymb,amsfonts,amsthm}
\usepackage{algorithmic}
\usepackage{graphicx}
\usepackage{textcomp}
\usepackage{xcolor}
\usepackage{txfonts}
\usepackage{listings}
\usepackage{enumitem}
\usepackage{mathtools}
\usepackage{gensymb}
\usepackage{comment}
\usepackage[breaklinks=true]{hyperref}
\usepackage{tkz-euclide} 
\usepackage{listings}
\usepackage{gvv}  
\usepackage{tikz}
\usepackage{circuitikz} 
\usepackage{caption}
\def\inputGnumericTable{}              
\usepackage[latin1]{inputenc}          
\usepackage{color}                    
\usepackage{array}                     
\usepackage{longtable}                 
\usepackage{calc}                     \usepackage{multirow}                  
\usepackage{hhline}                    
\usepackage{ifthen}                    
\usepackage{lscape}
\usepackage{amsmath}
\newtheorem{theorem}{Theorem}[section]
\newtheorem{problem}{Problem}
\newtheorem{proposition}{Proposition}[section]
\newtheorem{lemma}{Lemma}[section]
\newtheorem{corollary}[theorem]{Corollary}
\newtheorem{example}{Example}[section]
\newtheorem{definition}[problem]{Definition}
\newcommand{\BEQA}{\begin{eqnarray}}
\newcommand{\EEQA}{\end{eqnarray}}
\newcommand{\define}{\stackrel{\triangle}{=}}
\theoremstyle{remark}
\newtheorem{rem}{Remark}

%bibliographystyle(ieeetr}
\begin{document}
\bibliographystyle{ieeetr}
\vspace{3cm}

\title{Assignment-1}
\author{AI24BTECH11002 - K.AKSHAY TEJA}
\maketitle
\newpage
\bigskip

\renewcommand{\thefigure}{\theenumi}
\renewcommand{\thetable}{\theenumi}
\section{\textbf{SECTION A - JEE ADVANCED/ IIT-JEE}}

\textbf{E. Subjective Problems}
 \begin{enumerate}
    
\item   Evaluate the following $\int\frac{dx}{x^2(x^4 +1)^\frac{3}{4}} $ 
         \hfill  (1984 - 2 Marks)
\item   Evaluate the following $\int \sqrt{\frac{1 - \sqrt{x}}{1 + \sqrt{x}}}   dx$.\hfill
  (1985 - 2$\frac{1}{2}$ Marks)  


\item Evaluate: $\int$$[\frac{(cos2x)^{1/2}}{sin x})]dx$ \hfill  (1987 - 6 Marks)

\item   Evaluate {$\int(\sqrt{tan x}+\sqrt{cot x})dx$}      
  \hfill
(1989 - 3 Marks)
 

\item  Find the indefinite integral $\int$$(\frac{1}{\sqrt[3]{x} + \sqrt[4]{4}}+\frac{\ln (1+\sqrt[6]{x})}{\sqrt[3]{x}+\sqrt{x}}$  \hfill   (1992 - 4 Marks)

\item Find the indefinite integral \(\int \cos 2\theta \ln \frac{\cos \theta + \sin \theta}{\cos \theta - \sin \theta}   d\theta\) \hfill  (1994 - 5 Marks)


\item  Evaluate $\int\frac{(x + 1)}{x(1 + xe^x)^2}dx$  \hfill (1996 - 2 Marks)

\item Integrate $\int\frac{x^{3}+3x 2}{(x^{2}+ 1)^{2}(x + 1)}dx$ \hfill  (2001 - 5 Marks)

\item   Evaluate $\int \sin{-1} (\frac{2x + 2}{\sqrt{4x^{2}+8x+13}})dx$    \hfill    (2001 - 5 Marks)

\item  For any natural number m , evaluate ${\int(x^{3m} + x^{2m}+x^{m})(2x^{2m}+3x^{m}+6)^{1/m}dx,x>0}$   \hfill   (2002 - 5 Marks)
 
\end{enumerate}

\textbf{ H. Assertion \& Reason Type Questions}\\
 
\begin{enumerate}
\item   Let $F(x)$ be an indefinite integral of $\sin^2 x$\\

STATEMENT-1: The function $F(x)$ satisfies $F(x+\pi)=F(x)$for all real x. because\\
STATEMENT-2: $\sin^2$ (x + $\pi$ ) = $\sin^2 x$ for all real $x$.   \hfill     (2007 - 3 marks)
  
    \begin{enumerate}
    
 \item Statement-1 is True, Statement-2 is True; Statement-2 is a correct explanation for Statement-1.
\item Statement-1 is True, Statement-2 is True; Statement-2 is NOT a correct explanation for Statement-1
 \item Statement-1 is True, Statement-2 is False
\item Statement-1 is False, Statement-2 is True
\end{enumerate}
\end{enumerate} 

\large\large\section{\textbf{{SECTION-B Jee Main/AIEEE}}}\small\small

 \begin{enumerate}
\item If$\int$$\frac{\sin x}{\sin (x-\alpha}dx = Ax+B\log \sin (x-\alpha)+C,$then value of $(A,B)$ is    \hfill        [2004]
\begin{enumerate}
       \item { $(-\cos \alpha, \sin \alpha)$ }  
       \item  $(\cos \alpha,\sin \alpha)$
       \item    $(-\sin \alpha,\cos \alpha)$   
       \item  $(\sin \alpha,\cos \alpha)$ 
\end{enumerate}

\item   $\int\frac{dx}{\cos x-\sin x)}$ is equal to             \hfill [2004]
      \begin{enumerate}
\item  $\frac{1}{\sqrt{2}}\log \left|\tan \left(\frac{x}{2}+\frac{3\pi}{8}\right)\right|+C$

\item  $\frac{1}{\sqrt{2}}\log \left|\cot \left(\frac{x}{2}\right)\right|+C$ 

\item  {$\frac{1}{\sqrt{2}}\log \left|\tan \left(\frac{x}{2}-\frac{3\pi}{8}\right)\right|+C$}

\item  {$\frac{1}{\sqrt{2}}\log \left|\tan \left(\frac{x}{2}-\frac{\pi}{8}\right)\right|+C$}
\end{enumerate}
\item  $\int\left\{ \frac{(\log x-1)}{1+(\log x)^2}\right\}^2 dx$ is equal to     \hfill [2005]
 \begin{enumerate} 
\item  $\frac{\log x}{(\log x)^2 +1}+C$         
\item  $\frac{x}{x^2+1}+C$
\item  $\frac{xe^x}{1+x^2}+C$                       
\item  $\frac{x}{(\log x)^2+1}+C$
\end{enumerate}
\item $\int \frac{dx}{\cos x + \sqrt{3} \sin x}$ equals \hfill [2007]
  \begin{enumerate}
\item  $\log \tan \left(\frac{x}{2}+\frac{\pi}{12}\right)+C$
\item  $\log \tan \left(\frac{x}{2}-\frac{\pi}{12}\right)+C$
\item  $\frac{1}{2} \log \tan \left(\frac{x}{2}+\frac{\pi}{12}\right)+C$
\item  $\frac{1}{2} \log \tan \left(\frac{x}{2}+\frac{\pi}{12}\right)+C$
\end{enumerate}
 \end{enumerate}

\end{document}

