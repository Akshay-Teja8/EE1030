\documentclass[journal,9pt,onecolumn]{IEEEtran}
\usepackage[a5paper, margin=8mm]{geometry}
%\usepackage{lmodern} % Ensure lmodern is loaded for pdflatex
\usepackage{tfrupee} % Include tfrupee package

\setlength{\headheight}{1cm} % Set the height of the header box
\setlength{\headsep}{0mm}     % Set the distance between the header box and the top of the text

\usepackage{gvv-book}
\usepackage{gvv}
\usepackage{cite}
\usepackage{amsmath,amssymb,amsfonts,amsthm}
\usepackage{algorithmic}
\usepackage{graphicx}
\usepackage{textcomp}
\usepackage{xcolor}
\usepackage{txfonts}
\usepackage{listings}
\usepackage{enumitem}
\usepackage{mathtools}
\usepackage{gensymb}
\usepackage{comment}
\usepackage[breaklinks=true]{hyperref}
\usepackage{tkz-euclide} 
\usepackage{listings}
% \usepackage{gvv}                                        
\def\inputGnumericTable{}                                 
\usepackage[latin1]{inputenc}                                
\usepackage{color}                                            
\usepackage{array}                                            
\usepackage{longtable}                                       
\usepackage{calc}                                             
\usepackage{multirow}                                         
\usepackage{hhline}                                           
\usepackage{ifthen}                                           
\usepackage{lscape}
\begin{document}

\bibliographystyle{IEEEtran}
\vspace{3cm}
\title{2009-PH-13-24}
\author{AI24BTECH11002 - K. Akshay Teja}
\maketitle
 %\newpage
 \bigskip
{\let\newpage\relax\maketitle}

\renewcommand{\thefigure}{\theenumi}
\renewcommand{\thetable}{\theenumi}
\setlength{\intextsep}{10pt} % Space between text and floats

\numberwithin{equation}{enumi}
\numberwithin{figure}{enumi}
\renewcommand{\thetable}{\theenumi}

\begin{enumerate}


% Question 13
\item Identify which one is a first-order phase transition?
\begin{enumerate}
     \item A liquid to gas transition at its critical temperature
    \item A liquid to gas transition close to its triple point
    \item A paramagnetic to ferromagnetic transition in the absence of a magnetic field
    \item A metal to superconductor transition in the absence of a magnetic field
\end{enumerate}

% Question 14
\item Group I lists some physical phenomena while Group II gives some physical parameters. Match the phenomena with the corresponding parameters:
\begin{table}[h!]
 	\centering
 	\begin{tabular}{|c|c|}
	\hline
	Vector&Description\\
	\hline
	Vector A&  \(\hat{i} - 2 \hat{j} + 3 \hat{k}\)\\
	\hline
	Vector B& \(2\hat{i} +3 \hat{j} -4\hat{k}\)\\
	\hline
	Vector C& \(\hat{i} -3\hat{j} +\hat{k}\)\\
	\hline
\end{tabular}

 	\label{tab:Ph-2009}
 \end{table}
\begin{multicols}{2}
\begin{enumerate}
      \item P-4, Q-3, R-1, S-2
    \item P-3, Q-2, R-1, S-4
    \item P-2, Q-3, R-4, S-1
    \item P-1, Q-4, R-2, S-3
\end{enumerate}
\end{multicols}


% Question 15
\item  The separation between the first Stokes and corresponding anti-Stokes lines of the rotational Raman spectrum in terms of the rotational constant, B is
\begin{multicols}{4}
\begin{enumerate}
    \item 2B
    \item 4B
    \item 6B
    \item 12B
    \end{enumerate}
\end{multicols}


% Question 16
\item   A superconducting ring is cooled in the presence of a magnetic field below its critical temperature $\brak{T_c}$. The total magnetic flux that passes through the ring is
\begin{multicols}{4}    
\begin{enumerate}
    \item zero
    \item n$\frac{h}{2e}$
    \item $\frac{nh}{4\pi e}$
    \item $\frac{ne^2}{hc}$
\end{enumerate}
\end{multicols}


% Question 17
\item In a cubic crystal, atoms of mass $M_1$ lie on one set of planes and atoms of mass $M_2$ lie on planes interleaved between those of the first set. If $C$ is the force constant between nearest neighbor planes, the frequency of lattice vibrations for the optical phonon branch with wavevector $k = 0$ is

\begin{multicols}{4}
\begin{enumerate}
    \item $\sqrt{2C\brak{\frac{1}{M_1}+\frac{1}{M_2}}}$
    \item $\sqrt{C\brak{\frac{1}{2M_1}+\frac{1}{M_2}}}$
    \item $\sqrt{C\brak{\frac{1}{M_1}+\frac{1}{2M_2}}}$
    \item 0
\end{enumerate}
\end{multicols}


% Question 18
\item  In the quark model, which one of the following represents a proton?
This means there are
\begin{multicols}{4}    
\begin{enumerate}
    \item $udd$
    \item $uud$
    \item $a\overline{b}$
    \item $c\overline{e}$
\end{enumerate}
\end{multicols}

% Question 19
\item     
    \resizebox{0.3\textwidth}{!}{%
\begin{circuitikz}
\tikzstyle{every node}=[font=\LARGE]



\draw [short] (7.5,12.5) -- (12.75,12.5);
\draw  (10.25,14.5) circle (2cm);
\draw [->, >=Stealth] (10.25,14.5) -- (8.5,13.5);
\draw [short] (8.5,13.5) -- (7.75,13.5);
\node [font=\LARGE] at (10.25,15) {P};
\node [font=\LARGE] at (10.25,17.25) {Q};
\node [font=\LARGE] at (7.75,13.75) {r};
\node at (10.25,14.5) [circ] {};
\node at (10.25,16.5) [circ] {};
\end{circuitikz}
}%

    
 The circuit shown above
\begin{multicols}{2}    
\begin{enumerate}
    \item is a common-emitter amplifier
    \item uses a pop transistor
    \item is an oscillator
    \item has a voltage gain less than one
\end{enumerate}
\end{multicols}


% Question 20
\item Consider a nucleus with $N$ neutrons and $Z$ protons. If $m_n$, $m_p$, and $BE$ represent the mass of the neutron, the mass of the proton, and the binding energy of the nucleus respectively, and $c$ is the velocity of light in free space, the mass of the nucleus is given by
\begin{multicols}{4}
\begin{enumerate}
    \item $Nm_n + Zm_p$
    \item $Nm_p + Zm_n$
    \item $Nm_n + Zm_p + \frac{BE}{c^2}$
    \item $Nm_p + Zm_n + \frac{BE}{c^2}$
\end{enumerate}
\end{multicols}


% Question 21
\item The magnetic field \brak{in\,A\,m^{-1}} inside a long solid cylindrical conductor of radius $a = 0.1 \, \text{m}$ is $\overrightarrow{H} = \frac{10^4}{r} \sbrak{\frac{1}{\alpha^2} \sin\brak{\alpha r} - \frac{r}{\alpha} \cos\brak{\alpha r}} \hat{\phi}$ where $\alpha = \frac{\pi}{2a}$. What is the total current (in A) in the conductor?
\begin{multicols}{4}
\begin{enumerate}
    \item $\frac{\pi}{2a}$
    \item $\frac{800}{\pi}$
    \item $\frac{400}{\pi}$
    \item $\frac{300}{\pi}$
\end{enumerate}
\end{multicols}

% Question 22
\item Which one of the following current densities, $\overrightarrow{J}$, can generate the magnetic vector potential $\overrightarrow{A} = \brak{y^2 \hat{i} + x^2 \hat{j}}?$
\begin{multicols}{4}
\begin{enumerate}
    \item $\frac{2}{\mu_0} \brak{x \hat{i} + y \hat{j}}$
    \item $-\frac{2}{\mu_0} \brak{ \hat{i} +  \hat{j}}$
    \item $\frac{2}{\mu_0} \brak{ \hat{i}  - \hat{j}}$
    \item $\frac{2}{\mu_0} \brak{x \hat{i} - y \hat{j}}$
    
\end{enumerate}
\end{multicols}


% Question 11
\item The value of the integral $\int_{C} \frac{e^z}{z^2 - 3z + 2} \, dz$ where the contour $C$ is the circle $\abs{z} = \frac{3}{2}$ is
\begin{multicols}{4}    
\begin{enumerate}
    \item $2\pi i e$
    \item $\pi i e$
    \item $-2 \pi i e$
    \item $- \pi i e$
\end{enumerate}
\end{multicols}


% Question 12
\item In a non-conducting medium characterized by $\epsilon = \epsilon_0$, $\mu = \mu_0$, and conductivity $\sigma = 0$, the electric field \brak{in \,V \,m^{-1}} is given by $\overrightarrow{E} = 20 \sin \sbrak{10^8 t - kz } \hat{j}.$ The magnetic field, $\overrightarrow{H}$ \brak{in\,A\,m^{-1}}, is given by
\begin{multicols}{2}
\begin{enumerate}
    \item $20k \cos \sbrak{10^8 t - kz } \hat{i}$
    \item $\frac{20k}{10^8 \mu_0} \sin \sbrak{10^8 t - kz } \hat{j}$
    \item $-\frac{20k}{10^8 \mu_0} \sin \sbrak{10^8 t - kz } \hat{i}$
    \item $-20k \cos \sbrak{10^8 t - kz } \hat{j}$
\end{enumerate}
\end{multicols}


\end{enumerate}
\end{document}


