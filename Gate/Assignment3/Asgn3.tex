\documentclass[journal,9pt,onecolumn]{IEEEtran}
\usepackage[a5paper, margin=8mm]{geometry}
%\usepackage{lmodern} % Ensure lmodern is loaded for pdflatex
\usepackage{tfrupee} % Include tfrupee package

\setlength{\headheight}{1cm} % Set the height of the header box
\setlength{\headsep}{0mm}     % Set the distance between the header box and the top of the text

\usepackage{gvv-book}
\usepackage{gvv}
\usepackage{cite}
\usepackage{amsmath,amssymb,amsfonts,amsthm}
\usepackage{algorithmic}
\usepackage{graphicx}
\usepackage{textcomp}
\usepackage{xcolor}
\usepackage{txfonts}
\usepackage{listings}
\usepackage{enumitem}
\usepackage{mathtools}
\usepackage{gensymb}
\usepackage{comment}
\usepackage[breaklinks=true]{hyperref}
\usepackage{tkz-euclide} 
\usepackage{listings}
% \usepackage{gvv}                                        
\def\inputGnumericTable{}                                 
\usepackage[latin1]{inputenc}                                
\usepackage{color}                                            
\usepackage{array}                                            
\usepackage{longtable}                                       
\usepackage{calc}                                             
\usepackage{multirow}                                         
\usepackage{hhline}                                           
\usepackage{ifthen}                                           
\usepackage{lscape}
\begin{document}

\bibliographystyle{IEEEtran}
\vspace{3cm}
\title{2009-EE-13-24}
\author{AI24BTECH11002 - K. Akshay Teja}
\maketitle
 %\newpage
 \bigskip
{\let\newpage\relax\maketitle}

\renewcommand{\thefigure}{\theenumi}
\renewcommand{\thetable}{\theenumi}
\setlength{\intextsep}{10pt} % Space between text and floats

\numberwithin{equation}{enumi}
\numberwithin{figure}{enumi}
\renewcommand{\thetable}{\theenumi}

\begin{enumerate}

\setcounter{enumi}{12}
% Question 13
\item The complete set of only those Logic Gates designated as Universal Gates is
\begin{multicols}{2}
\begin{enumerate}
     \item NOT, OR and AND Gates
    \item XNOR, NOR and NAND Gates
    \item NOR and NAND Gates
    \item XOR, NOR and NAND Gates
\end{enumerate}
\end{multicols}

% Question 14
\item The single phase, 50 Hz , iron core transformer in the circuit has both the vertical arms of cross sectional area $20$ cm$^{2}$ and both the horizontal arms of cross sectional area $10$ cm$^{2}$. If the two windings shown were wound instead on opposite horizontal arms, the mutual inductance will
\begin{center}
    \input{figs/fig.tex}
\end{center}
\begin{multicols}{4}
\begin{enumerate}
     \item double
    \item remain same
    \item be halved
    \item become one quarter
\end{enumerate}
\end{multicols}


% Question 15
\item A 3-phase squirrel cage induction motor supplied from a balanced 3-phase source drives a mechanical load. The torque-speed characteristics of the motor (solid curve) and of the load (dotted curve) are shown. Of the two equilibrium points A and B, which of the following options correctly describes the stability of A and B?
\begin{center}
    \input{figs/fig2.tex}
\end{center}
\begin{multicols}{2}
    \begin{enumerate}
    \item A is stable, B is unstable
    \item A is unstable, B is stable
    \item Both are stable
    \item Both are unstable
    \end{enumerate}
\end{multicols}


% Question 16
\item  An SCR is considered to be a semi-controlled device because
\begin{enumerate}
    \item it can be turned OFF but not ON with a gate pulse
    \item it conducts only during one half-cycle of an alternating current wave
    \item it can be turned ON but not OFF with a gate pulse
    \item it can be turned ON only during one half-cycle of an alternating voltage wave
\end{enumerate}


% Question 17
\item The polar plot of an open loop stable system is shown below. The closed loop system is
\begin{center}
    \input{figs/fig3.tex}
\end{center}
\begin{multicols}{2}
\begin{enumerate}
    \item always stable
    \item marginally stable
    \item unstable with one pole on the RH s-plane
    \item unstable with two poles on the RH s-plane
\end{enumerate}
\end{multicols}


% Question 18
\item  The first two rows of Routh's tabulation of a third order equation are as follows.

\begin{center}
\begin{tabular}{lll}
$s^{3}$ & 2 & 2 \\
$s^{2}$ & 4 & 4 \\
\end{tabular}
\end{center}
This means there are
\begin{enumerate}
    \item two roots at $s= \pm j$ and one root in right half s-plane
    \item  two roots at $s= \pm j 2$ and one root in left half s-plane
    \item  two roots at $s= \pm j 2$ and one root in right half s-plane
    \item  two roots at $s= \pm j$ and one root in left half s-plane
\end{enumerate}



% Question 19
\item The asymptotic approximation of the log-magnitude $v s$ frequency plot of a system containing only real poles and zeros is shown. Its transfer function is
\begin{center}
    \input{figs/fig4.tex}
\end{center}
\begin{multicols}{4}    
\begin{enumerate}
    \item $\frac{10\brak{s+5}}{s\brak{s+2}\brak{s+25}}$
    \item $\frac{1000\brak{s+5}}{s^{2}\brak{s+2}\brak{s+25}}$
    \item $\frac{100\brak{s+5}}{s\brak{s+2}\brak{s+25}}$
    \item $\frac{80\brak{s+5}}{s^{2}\brak{s+2}\brak{s+25}}$
\end{enumerate}
\end{multicols}


% Question 20
\item The trace and determinant of a $2 \times 2$ matrix are known to be -2 and -35 respectively. Its eigenvalues are
\begin{multicols}{4}
\begin{enumerate}
    \item $-30$ and $-5$
    \item $-37$ and $-1$
    \item $-7$ and $5$
    \item $17.5$ and $-2$
\end{enumerate}
\end{multicols}


% Question 21
\item The following circuit has $R=10 k \Omega, C=10 \mu  F$ The input voltage is a sinusoid at 50 Hz with an rms value of 10 V . Under ideal conditions, the current $i_{S}$ from the source is
\begin{center}
    \input{figs/fig5.tex}
\end{center}
\begin{multicols}{2}
\begin{enumerate}
    \item $10 \pi   mA$ leading by $90\degree$
    \item $20 \pi   mA$ leading by $90\degree$
    \item $10   mA$ leading by $90\degree$
    \item $10 \pi   mA$ lagging by $90\degree$
\end{enumerate}
\end{multicols}

% Question 22
\item In the figure shown, all elements used are ideal. For time $t < 0$, $S_1$ remained closed and $S_2$ open. At $t=0$, $S_1$ is opened and $S_2$ is closed. If the voltage $V_{c2}$ across the capacitor $C_2$ at $t=0$ is zero, the voltage across the capacitor combination at $t=0^{+}$ will be
\begin{center}
    \input{figs/fig6.tex}
\end{center}
\begin{multicols}{4}
\begin{enumerate}
    \item $1$ V
    \item $2$ V
    \item $1.5$ V
    \item $3$ V
\end{enumerate}
\end{multicols}


% Question 23
\item Transformer and emitter follower can both be used for impedance matching at the output of an audio amplifier. The basic relationship between the input power $P_{in}$ and output power $P_{out}$ in both cases is
\begin{enumerate}
    \item P$_{in} = $ P$_{out}$ for both transformer and emitter follower
    \item P$_{in} > $ P$_{out}$ for both transformer and emitter follower
    \item P$_{in} < $ P$_{out}$ for transformer and P$_{in} = $ P$_{out}$ for emitter follower
    \item P$_{in} = $ P$_{out}$ for transformer and P$_{in} < $ P$_{out}$ for emitter follower
\end{enumerate}


% Question 24
\item The equivalent capacitance of the input loop of the circuit shown is
\begin{center}
    \input{figs/figure.tex}
\end{center}
\begin{multicols}{4}
\begin{enumerate}
    \item $2~\mu$F
    \item $100~\mu$F
    \item $200~\mu$F
    \item $4~\mu$F
\end{enumerate}
\end{multicols}

\end{enumerate}
\end{document}


