\documentclass[journal,9pt,onecolumn]{IEEEtran}
\usepackage[a5paper, margin=8mm]{geometry}
%\usepackage{lmodern} % Ensure lmodern is loaded for pdflatex
\usepackage{tfrupee} % Include tfrupee package

\setlength{\headheight}{1cm} % Set the height of the header box
\setlength{\headsep}{0mm}     % Set the distance between the header box and the top of the text

\usepackage{gvv-book}
\usepackage{gvv}
\usepackage{cite}
\usepackage{amsmath,amssymb,amsfonts,amsthm}
\usepackage{algorithmic}
\usepackage{graphicx}
\usepackage{textcomp}
\usepackage{xcolor}
\usepackage{txfonts}
\usepackage{listings}
\usepackage{enumitem}
\usepackage{mathtools}
\usepackage{gensymb}
\usepackage{comment}
\usepackage[breaklinks=true]{hyperref}
\usepackage{tkz-euclide} 
\usepackage{listings}
% \usepackage{gvv}                                        
\def\inputGnumericTable{}                                 
\usepackage[latin1]{inputenc}                                
\usepackage{color}                                            
\usepackage{array}                                            
\usepackage{longtable}                                       
\usepackage{calc}                                             
\usepackage{multirow}                                         
\usepackage{hhline}                                           
\usepackage{ifthen}                                           
\usepackage{lscape}
\begin{document}

\bibliographystyle{IEEEtran}
\vspace{3cm}
\title{2018-PH-53-65}
\author{AI24BTECH11002 - K. Akshay Teja}
\maketitle
 %\newpage
 \bigskip
{\let\newpage\relax\maketitle}

\renewcommand{\thefigure}{\theenumi}
\renewcommand{\thetable}{\theenumi}
\setlength{\intextsep}{10pt} % Space between text and floats

\numberwithin{equation}{enumi}
\numberwithin{figure}{enumi}
\renewcommand{\thetable}{\theenumi}


\begin{enumerate}
\setcounter{enumi}{52}
% Question 1
\item For the transformation $$Q = \sqrt{2q} \, e^{-1+2\alpha} \cos p,\, P = \sqrt{2q} \, e^{\alpha-1} \sin p$$

(where $\alpha$ is a constant) to be canonical, the value of $\alpha$ is \_\_\_\_\_.
\hfill \brak{2018}

% Question 2
\item Given $$\frac{d^2 f\brak{x}}{dx^2} - 2 \frac{df\brak{x}}{dx} + f\brak{x} = 0,$$
and boundary conditions $f\brak{0} = 1$ and $f\brak{1} = 0$, the value of $f\brak{0.5}$ is \_\_\_\_\_ (up to two decimal places).\hfill \brak{2018}



% Question 3
\item The absolute value of the integral $$\int \frac{5z^3 + 3z^2}{z^2 - 4} \, dz,$$
over the circle $\abs{z - 1.5} = 1$ in the complex plane, is \_\_\_\_\_\_ (up to two decimal places).\hfill \brak{2018}


% Question 4
\item A uniform circular disc of mass $m$ and radius $R$ is rotating with angular speed $\omega$ about an axis passing through its center and making an angle $\theta = 30\degree$ with the axis of the disc. If the kinetic energy of the disc is $\alpha m \omega^2 R^2$, the value of $\alpha$ is \_\_\_\_\_\_ (up to 2 decimal places).\hfill \brak{2018}
\begin{center}
    \resizebox{0.3\textwidth}{!}{%
\begin{circuitikz}
\tikzstyle{every node}=[font=\LARGE]



\draw [short] (7.5,12.5) -- (12.75,12.5);
\draw  (10.25,14.5) circle (2cm);
\draw [->, >=Stealth] (10.25,14.5) -- (8.5,13.5);
\draw [short] (8.5,13.5) -- (7.75,13.5);
\node [font=\LARGE] at (10.25,15) {P};
\node [font=\LARGE] at (10.25,17.25) {Q};
\node [font=\LARGE] at (7.75,13.75) {r};
\node at (10.25,14.5) [circ] {};
\node at (10.25,16.5) [circ] {};
\end{circuitikz}
}%


\end{center}

% Question 5
\item The ground state energy of a particle of mass $m$ in an infinite potential well is $E_0$. It changes to $E_0 \brak{1 + \alpha \times 10^{-3}}$ when there is a small potential bump of height $V_0 = \frac{\pi^2 \hbar^2}{50 m L^2}$ and width $a = \frac{L}{100}$, as shown in the figure. The value of $\alpha$ is \_\_\_\_\_\_ (up to two decimal places).\hfill \brak{2018}
\begin{center}
    \resizebox{0.3\textwidth}{!}{%
\begin{circuitikz}
\tikzstyle{every node}=[font=\LARGE]
\draw [short] (7.75,15) -- (7.75,11);
\draw [short] (7.75,11) -- (9.75,11);
\draw [short] (10.75,11) -- (12.75,11);
\draw [short] (12.75,11) -- (12.75,15);
\draw [short] (9.75,11) -- (9.75,12);
\draw [short] (9.75,12) -- (10.75,12);
\draw [short] (10.75,12) -- (10.75,11);
\draw [->, >=Stealth] (5.5,13.5) -- (5.5,15);
\draw [->, >=Stealth] (5.5,13.5) -- (7,13.5);
\node [font=\LARGE] at (6.25,15) {$V\brak{x}$};
\node [font=\LARGE] at (7,13) {x};
\node [font=\LARGE] at (10.25,12.5) {a};
\node [font=\LARGE] at (10.25,10.5) {L};
\node [font=\LARGE] at (11.5,11.5) {$V_0$};
\draw [<->, >=Stealth] (11,12) -- (11,11);
\draw [->, >=Stealth] (10.75,10.5) -- (12.75,10.5);
\draw [->, >=Stealth] (9.75,10.5) -- (7.75,10.5);
\end{circuitikz}
}%


\end{center}

% Question 6
\item An electromagnetic plane wave is propagating with an intensity $I = 1.0 \times 10^5$ Wm$^2$ in a medium with $\varepsilon = 3\varepsilon_0$ and $\mu = \mu_0$. The amplitude of the electric field inside the medium is \_\_\_\_\_\_$ \times 10^3$ Vm$^{-1}$ (up to one decimal place).\hfill \brak{2018}
$$
\brak{\varepsilon_0 = 8.85 \times 10^{-12} {C}^2N^{-1}m^2, \, \mu_0 = 4\pi \times 10^{-7} \, N\cdot A^{-2}, \, c = 3 \times 10^8 \, ms^{-1}}
$$

% Question 7
\item A microcanonical ensemble consists of 12 atoms with each taking either energy 0 state or energy $\varepsilon$ state. Both states are non-degenerate. If the total energy of this ensemble is $4\varepsilon$, its entropy will be \_\_\_\_\_\_\_$ k_B$ (up to one decimal place), where $k_g$ is the Boltzmann constant.\hfill \brak{2018}

% Question 8
\item A two-state quantum system has energy eigenvalues $\pm \varepsilon$ corresponding to the normalized states $|\psi \pm\rangle$. At time $t = 0$, the system is in quantum state $\frac{1}{\sqrt{2}}\brak{|\psi_+\rangle + |\psi_-\rangle}$. The probability that the system will be in the same state at $t = \frac{h}{\brak{6\varepsilon}}$ is \_\_\_\_\_ (up to two decimal places).\hfill \brak{2018}


    
% Question 9
\item An air-conditioner maintains the room temperature at $27\degree$ C while the outside temperature is $47\degree$ C. The heat conducted through the walls of the room from outside to inside due to temperature difference is $7000$ W. The minimum work done by the compressor of the air-conditioner per unit time is \_\_\_\_\_ W.\hfill \brak{2018}

% Question 10
\item Two solid spheres $A$ and $B$ have the same emissivity. The radius of $A$ is four times the radius of $B$, and the temperature of $A$ is twice the temperature of $B$. The ratio of the rate of heat radiated from $A$ to that from $B$ is \_\_\_\_\_.\hfill \brak{2018}


% Question 11
\item The partition function of an ensemble at a temperature $T$ is: $$Z = \brak{ 2 \cosh\brak{\frac{\varepsilon}{k_B T}} }^N$$

where $k_B$ is the Boltzmann constant. The heat capacity of this ensemble at $T = \frac{\varepsilon}{k_B}$ is $X N k_B$, where the value of $X$ is \_\_\_\_\_ (up to two decimal places).\hfill \brak{2018}



% Question 12
\item An atom in its singlet state is subjected to a magnetic field. The Zeeman splitting of its $650$ nm spectral line is $0.03$ nm. The magnitude of the field is \_\_\_\_\_ Tesla (up to two decimal places). 
$$\brak{e = 1.60 \times 10^{-19} \,C,\, m_e = 9.11 \times 10^{-31} \, kg, \, c = 3.0 \times 10^8 \, ms^{-1}}$$
\hfill \brak{2018}

% Question 13
\item The quantum effects in an ideal gas become important below a certain temperature $T_0$ when the de Broglie wavelength corresponding to the root mean square thermal speed becomes equal to the inter-atomic separation. For such a gas of atoms of mass $2 \times 10^{-26}$ kg and number density $6.4 \times 10^{25}$ m$^{-3}$, $T_Q =$ \_\_\_\_\_\_$\times 10^{-3}$ K (up to one decimal place).  \hfill \brak{2018}

$$\brak{k_B = 1.38 \times 10^{-23} \, J/K,\, h = 6.6 \times 10^{-34} \, J\cdot s}$$
\end{enumerate}
\end{document}
