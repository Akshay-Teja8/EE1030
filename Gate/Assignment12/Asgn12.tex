\documentclass[journal,9pt,onecolumn]{IEEEtran}
\usepackage[a5paper, margin=8mm]{geometry}
%\usepackage{lmodern} % Ensure lmodern is loaded for pdflatex
\usepackage{tfrupee} % Include tfrupee package

\setlength{\headheight}{1cm} % Set the height of the header box
\setlength{\headsep}{0mm}     % Set the distance between the header box and the top of the text

\usepackage{gvv-book}
\usepackage{gvv}
\usepackage{cite}
\usepackage{amsmath,amssymb,amsfonts,amsthm}
\usepackage{algorithmic}
\usepackage{graphicx}
\usepackage{textcomp}
\usepackage{xcolor}
\usepackage{txfonts}
\usepackage{listings}
\usepackage{enumitem}
\usepackage{mathtools}
\usepackage{gensymb}
\usepackage{comment}
\usepackage[breaklinks=true]{hyperref}
\usepackage{tkz-euclide} 
\usepackage{listings}
% \usepackage{gvv}                                        
\def\inputGnumericTable{}                                 
\usepackage[latin1]{inputenc}                                
\usepackage{color}                                            
\usepackage{array}                                            
\usepackage{longtable}                                       
\usepackage{calc}                                             
\usepackage{multirow}                                         
\usepackage{hhline}                                           
\usepackage{ifthen}                                           
\usepackage{lscape}
\begin{document}

\bibliographystyle{IEEEtran}
\vspace{3cm}
\title{2014-PH-27-39}
\author{AI24BTECH11002 - K. Akshay Teja}
\maketitle
 %\newpage
 \bigskip
{\let\newpage\relax\maketitle}

\renewcommand{\thefigure}{\theenumi}
\renewcommand{\thetable}{\theenumi}
\setlength{\intextsep}{10pt} % Space between text and floats

\numberwithin{equation}{enumi}
\numberwithin{figure}{enumi}
\renewcommand{\thetable}{\theenumi}


\begin{enumerate}
\setcounter{enumi}{26}
% Question 27
\item An electron in the ground state of the hydrogen atom has the wave function: $$\psi \brak{\overrightarrow{r}} = \frac{1}{\sqrt{\pi a_0^3}} e^{-\brak{\frac{r}{a_0}}}$$ where $a_0$ is a constant. The expectation value of the operator $\hat{Q} = z^2 - r^2$, where $z = r \cos \theta$, is\\(Hint: $\int_0^\infty e^{-ar} r^ndr=\frac{\Gamma\brak{n}}{\alpha^{n+1}}=\frac{\brak{n-1}!}{\alpha^{n+!}}$
\begin{multicols}{4}
\begin{enumerate}
    \item $-\frac{a_0^2}{2}$
    \item $-a_0^2$
    \item $-\frac{3a_0^2}{2}$
    \item $-2a_0^2$
\end{enumerate}
\end{multicols}

% Question 28
\item For Nickel, the number density is $8 \times 10^{23}$ atoms/cm$^3$ and the electronic configuration is \\$1 s^2 2 s^2 2 p^6 3 s^2 3 p^6 3 d^8 4 s^2$. The value of the saturation magnetization of Nickel in its ferromagnetic state is:

(Given the value of Bohr magneton $\mu_B = 9.21 \times 10^{-21}$Am$^2$)


% Question 29
\item A particle of mass $m$ is in a potential given by $$V \brak{r} = -\frac{a}{r} + \frac{a r_0^2}{3 r^3}$$ where $a$ and $r_0$ are positive constants. When disturbed slightly from its stable equilibrium position, it undergoes simple harmonic oscillation. The time period of oscillation is:

\begin{multicols}{4}
\begin{enumerate}
    \item $2\pi \sqrt{\frac{m r_0^3}{2a}}$
    \item $2\pi \sqrt{\frac{m r_0^3}{a}}$
    \item $2\pi \sqrt{\frac{2 m r_0^3}{a}}$
    \item $4\pi \sqrt{\frac{m r_0^3}{a}}$
\end{enumerate}
\end{multicols}

% Question 30
\item The donor concentration in a sample of $n$-type silicon is increased by a factor of 100. The shift in the position of the Fermi level at 300K, assuming the sample to be non-degenerate is \\  
($k_B T = 25$ meV at 300K)




% Question 31
\item A particle of mass $m$ is subjected to a potential: $$V \brak{x, y} = \frac{1}{2} m \omega^2  \brak{x^2 + y^2}, -\infty \leq x \leq \infty, -\infty \leq y \leq \infty$$. The state with energy $4\hbar \omega$ is $g$-fold degenerate. The value of $g$ is:


% Question 32
\item A hydrogen atom is in the state: $$\psi = \sqrt{\frac{8}{21}} \psi_{200} - \sqrt{\frac{3}{7}} \psi_{310} + \sqrt{\frac{4}{21}} \psi_{321},$$ where $n$, $l$, $m$ in $\psi_{n l m}$ denote the principal, orbital, and magnetic quantum numbers, respectively. If $\overrightarrow{L}$ is the angular momentum operator, the average value of $ L^2$ is \,\,\,\,\,$h^2$

    
% Question 33
\item A planet of mass $m$ moves in a circular orbit of radius $r_0$ in the gravitational potential $V \brak{r} = -\frac{k}{r}$, where $k$ is a positive constant. The orbital angular momentum of the planet is:

\begin{multicols}{4}
\begin{enumerate}
    \item $2 r_0 k m$
    \item $\sqrt{2 r_0 k m}$
    \item $r_0 k m$
    \item $\sqrt{r_0 k m}$
\end{enumerate}
\end{multicols}

    
% Question 34
\item The moment of inertia of a rigid diatomic molecule A is 6 times that of another rigid diatomic molecule B. If the rotational energies of the two molecules are equal, then the corresponding values of the rotational quantum numbers $J_A$ and $J_B$ are:

\begin{multicols}{2}
\begin{enumerate}
    \item $J_A = 2$, $J_B = 1$
    \item $J_A = 3$, $J_B = 1$
    \item $J_A = 5$, $J_B = 0$
    \item $J_A = 6$, $J_B = 1$
\end{enumerate}
\end{multicols}



% Question 35
\item The value of the integral:$$\oint_C \frac{z^2}{e^z + 1} \, dz,$$
where $C$ is the circle $\abs{z} = 4$, is:

\begin{multicols}{4}
\begin{enumerate}
    \item $2 \pi i$
    \item $2 \pi^2 i$
    \item $4 \pi^3 i$
    \item $4 \pi^2 i$
\end{enumerate}
\end{multicols}

% Question 36
\item A ray of light inside Region 1 in the $xy$-plane is incident at the semicircular boundary that carries no free charges. The electric field at the point $P\brak{r, \pi/4}$ in plane polar coordinates is $\overrightarrow{E_1} = 7 e_0 \hat{e}_r - 3 e_0 \hat{e}_{\theta}$where $\hat{e}_r$ and $\hat{e}_{\theta}$ are the unit vectors. The emerging ray in Region 2 has the electric field $\overrightarrow{E_2}$ parallel to the $x$-axis. If $\epsilon_1$ and $\epsilon_2$ are the dielectric constants of Region 1 and Region 2 respectively, then $\frac{\epsilon_1}{\epsilon_2}$ is
\begin{center}
    \resizebox{0.3\textwidth}{!}{%
\begin{circuitikz}
\tikzstyle{every node}=[font=\LARGE]



\draw [short] (7.5,12.5) -- (12.75,12.5);
\draw  (10.25,14.5) circle (2cm);
\draw [->, >=Stealth] (10.25,14.5) -- (8.5,13.5);
\draw [short] (8.5,13.5) -- (7.75,13.5);
\node [font=\LARGE] at (10.25,15) {P};
\node [font=\LARGE] at (10.25,17.25) {Q};
\node [font=\LARGE] at (7.75,13.75) {r};
\node at (10.25,14.5) [circ] {};
\node at (10.25,16.5) [circ] {};
\end{circuitikz}
}%


\end{center}

% Question 37
\item The solution of the differential equation: $$\frac{d^2 y}{dt^2} - y = 0$$subject to the boundary conditions $y\brak{0} = 1$ and $y\brak{\infty} = 0$, is:

\begin{multicols}{4}
\begin{enumerate}
    \item $\cos t + \sin t$
    \item $\cosh t + \sinh t$
    \item $\cos t - \sin t$
    \item $\cosh t - \sinh t$
\end{enumerate}
\end{multicols}


% Question 38
\item Given that the linear transformation of a generalized coordinate $q$ and the corresponding momentum $p$,  
$$
Q = q + 4a p
$$
$$
P = q + 2p
$$
is canonical, the value of the constant $a$ is:




    
% Question 39
\item The value of the magnetic field required to maintain non-relativistic protons of energy 1 MeV in a circular orbit of radius 100 mm is:  

(Given: $m_p = 1.67 \times 10^{-27}$ kg, e = 1.6 $\times 10^{-19}$ C)
\end{enumerate}
\end{document}


