\documentclass[journal,9pt,onecolumn]{IEEEtran}
\usepackage[a5paper, margin=8mm]{geometry}
%\usepackage{lmodern} % Ensure lmodern is loaded for pdflatex
\usepackage{tfrupee} % Include tfrupee package

\setlength{\headheight}{1cm} % Set the height of the header box
\setlength{\headsep}{0mm}     % Set the distance between the header box and the top of the text

\usepackage{gvv-book}
\usepackage{gvv}
\usepackage{cite}
\usepackage{amsmath,amssymb,amsfonts,amsthm}
\usepackage{algorithmic}
\usepackage{graphicx}
\usepackage{textcomp}
\usepackage{xcolor}
\usepackage{txfonts}
\usepackage{listings}
\usepackage{enumitem}
\usepackage{mathtools}
\usepackage{gensymb}
\usepackage{comment}
\usepackage[breaklinks=true]{hyperref}
\usepackage{tkz-euclide} 
\usepackage{listings}
% \usepackage{gvv}                                        
\def\inputGnumericTable{}                                 
\usepackage[latin1]{inputenc}                                
\usepackage{color}                                            
\usepackage{array}                                            
\usepackage{longtable}                                       
\usepackage{calc}                                             
\usepackage{multirow}                                         
\usepackage{hhline}                                           
\usepackage{ifthen}                                           
\usepackage{lscape}
\begin{document}

\bibliographystyle{IEEEtran}
\vspace{3cm}
\title{2009-XE-1-12}
\author{AI24BTECH11002 - K. Akshay Teja}
\maketitle
 %\newpage
 \bigskip
{\let\newpage\relax\maketitle}

\renewcommand{\thefigure}{\theenumi}
\renewcommand{\thetable}{\theenumi}
\setlength{\intextsep}{10pt} % Space between text and floats

\numberwithin{equation}{enumi}
\numberwithin{figure}{enumi}
\renewcommand{\thetable}{\theenumi}

\begin{enumerate}


% Question 1
\item Let A and B be two similar square matrices of order two. If 1 and -2 are the eigenvalues of A, then the Trace of B is
\begin{multicols}{4}    
\begin{enumerate}
     \item -2
     \item -1
     \item 1
     \item 2
\end{enumerate}
\end{multicols}

% Question 2
\item The root of ax + b = 0 (a, b constants), can be found by the Newton-Raphson method with a minimum of
\begin{multicols}{2}
\begin{enumerate}
    \item 1 iteration
    \item 2 iterations
    \item 3 iterations
    \item an undeterminable number of iterations
\end{enumerate}
\end{multicols}


% Question 3
\item  The solution $u\brak{x,t}$ of the one-dimensional heat equation, 
$$\frac{\partial u}{\partial t} = c^2 \frac{\partial^2 u}{\partial x^2},x\in \mathbb{R}$$ with a Gaussian initial condition,
\begin{enumerate}
    \item  travels with finite constant wave-speed
    \item  travels with finite variable wave-speed
    \item  spreads in both directions, with the magnitude of the peak increasing with time
    \item  spreads in both directions, with the magnitude of the peak decreasing with time
\end{enumerate}


% Question 4
\item  Let $C$ be the boundary of the square given by $0 \leq x \leq 1$, $0 \leq y \leq 1$. Then $\oint_C \brak{x \, dy - y \, dx}$ equals  
\begin{multicols}{4}    
\begin{enumerate}
    \item -2
    \item 0
    \item 1
    \item 2
\end{enumerate}
\end{multicols}


% Question 5
\item Let the eigenvalues of a square matrix A of order two be 1 and 2. The corresponding eigenvectors are $\begin{pmatrix}0.6\\ 0.8 \end{pmatrix}$ and $\begin{pmatrix}0.8\\ -0.6\end{pmatrix}$, respectively. Then, the element A\brak{2,2} is
\begin{multicols}{4}
\begin{enumerate}
    \item -0.48
    \item 0.48
    \item 1.36
    \item 1.64
\end{enumerate}
\end{multicols}


% Question 6
\item  Let $y_1\brak{x}$ and $y_2\brak{x}$ be two linearly independent solutions of $$\frac{d^2y}{dx^2} + \frac{6}{x} \frac{dy}{dx} + q\brak{x}y = 0,  x \in \brak{1,3}$$ where $q\brak{x}$ is a continuous function in $\brak{1,3}$. If the Wronskian of $y_1\brak{x}$ and $y_2\brak{x}$ at $x = 1$, denoted by $W\brak{y_1,y_2}\brak{1}$, is 1, then $W\brak{y_1,y_2}\brak{2}$ is
\begin{multicols}{4}    
\begin{enumerate}
    \item $\frac{1}{2^6}$
    \item $\frac{1}{2^3}$
    \item $\frac{1}{2}$
    \item 2
\end{enumerate}
\end{multicols}

% Question 7
\item    Simpson's $\frac{1}{3}$ rule applied to $\int_{-1}^{1} \brak{3x^2 + 5} \, dx$ with sub-interval $h = 1$, will give
\begin{multicols}{2}    
\begin{enumerate}
    \item  the exact result
    \item  error between 0.01\% to 0.1\%
    \item  error between 0.1\% to 1.0\%
    \item  error $> 1.0\%$
\end{enumerate}
\end{multicols}


% Question 8
\item The probability that a six-sided dice is thrown $n$ times without giving a '6', even once, is
\begin{multicols}{4}    
\begin{enumerate}
    \item $\brak{ \frac{5}{6} }^n$
    \item $\frac{5}{6} \brak{ \frac{1}{6} }^n$
    \item $\frac{\brak{n-1}!}{n!} \brak{ \frac{1}{6} }^n$
    \item $1 - \frac{\brak{n-1}!}{n!} \brak{ \frac{5}{6} }^n$
\end{enumerate}
\end{multicols}


% Question 9
\item If a complex function $f\brak{z} = u\brak{x,y} + iv\brak{x,y}$ is analytic, then

\begin{multicols}{4}
\begin{enumerate}
    \item $i \frac{\partial u}{\partial x} + \frac{\partial v}{\partial x} = \frac{\partial u}{\partial y} + i \frac{\partial v}{\partial y}$
    \item $i \frac{\partial u}{\partial x} + \frac{\partial v}{\partial x} = -\frac{\partial u}{\partial y} - i \frac{\partial v}{\partial y}$
    \item $ \frac{\partial u}{\partial x} + i\frac{\partial v}{\partial x} = -i\frac{\partial u}{\partial y} + \frac{\partial v}{\partial y}$
    \item $\frac{\partial u}{\partial x} + i \frac{\partial v}{\partial x} = i \frac{\partial u}{\partial y} - \frac{\partial v}{\partial y}$
\end{enumerate}
\end{multicols}

% Question 10
\item Let $\overrightarrow{u} = -\omega y\hat{i} + \omega x\hat{j}$ and $\overrightarrow{v} = \omega z\hat{j} - \omega y\hat{k}$ be two given vectors, where $\omega$ is a constant. Then $div\brak{\overrightarrow{u} \times \overrightarrow{v}}$ equals

\begin{multicols}{4}
\begin{enumerate}
     \item $0$
    \item $2\omega^2 y$
    \item $4\omega^2 y$
    \item $-4\omega^2 y$
\end{enumerate}
\end{multicols}


% Question 11
\item The infinite series $\sum_{m=1}^{\infty} \frac{\brak{-1}^m x^2}{\brak{1 + x^2}^m}$ is
\begin{multicols}{2}    
\begin{enumerate}
    \item Divergent for all $x$
    \item Convergent only for $x \geq 1$
    \item Convergent for all $x$
    \item Divergent only for $-1 \leq x \leq 1$
\end{enumerate}
\end{multicols}


% Question 12
\item Let $f\brak{x}$ be continuous and satisfy $m \leq f\brak{x} \leq M$ in $1 \leq x \leq 10$. Then, $\mu = \frac{\int_{1}^{10} \brak{f\brak{x} x^2} \, dx}{\int_{1}^{10} \brak{x^2} \, dx}$ satisfies
\begin{multicols}{4}
\begin{enumerate}
     \item $\mu \leq 333m$
    \item $333\mu \geq M$
    \item $m \leq \mu \leq M$
    \item $m \leq \mu \leq \frac{M}{333}$
\end{enumerate}
\end{multicols}

\end{enumerate}
\end{document}
