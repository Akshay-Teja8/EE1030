
\documentclass[journal,9pt,onecolumn]{IEEEtran}
\usepackage[a5paper, margin=8mm]{geometry}
%\usepackage{lmodern} % Ensure lmodern is loaded for pdflatex
\usepackage{tfrupee} % Include tfrupee package

\setlength{\headheight}{1cm} % Set the height of the header box
\setlength{\headsep}{0mm}     % Set the distance between the header box and the top of the text

\usepackage{gvv-book}
\usepackage{gvv}
\usepackage{cite}
\usepackage{amsmath,amssymb,amsfonts,amsthm}
\usepackage{algorithmic}
\usepackage{graphicx}
\usepackage{textcomp}
\usepackage{xcolor}
\usepackage{txfonts}
\usepackage{listings}
\usepackage{enumitem}
\usepackage{mathtools}
\usepackage{gensymb}
\usepackage{comment}
\usepackage[breaklinks=true]{hyperref}
\usepackage{tkz-euclide} 
\usepackage{listings}
% \usepackage{gvv}                                        
\def\inputGnumericTable{}                                 
\usepackage[latin1]{inputenc}                                
\usepackage{color}                                            
\usepackage{array}                                            
\usepackage{longtable}                                       
\usepackage{calc}                                             
\usepackage{multirow}                                         
\usepackage{hhline}                                           
\usepackage{ifthen}                                           
\usepackage{lscape}
\begin{document}

\bibliographystyle{IEEEtran}
\vspace{3cm}
\title{2008-PH-1-17}
\author{AI24BTECH11002 - K. Akshay Teja}
\maketitle
 %\newpage
 \bigskip
{\let\newpage\relax\maketitle}

\renewcommand{\thefigure}{\theenumi}
\renewcommand{\thetable}{\theenumi}
\setlength{\intextsep}{10pt} % Space between text and floats

\numberwithin{equation}{enumi}
\numberwithin{figure}{enumi}
\renewcommand{\thetable}{\theenumi}

\begin{enumerate}


% Question 1
\item For arbitrary matrices $E$, $F$, $G$, and $H$, if $EF - FE = 0$, then $\text{Trace}\brak{EFGH}$ is equal to:
\begin{multicols}{2}
\begin{enumerate}
    \item Trace\brak{HGFE}
    \item $Trace\brak{E} Trace\brak{F}Trace\brak{G}Trace\brak{H}$
    \item Trace\brak{GFEH}
    \item Trace\brak{EGHF}
\end{enumerate}
\end{multicols}

% Question 2
\item A unitary matrix $\begin{pmatrix} ae^{i\alpha} & b \\ ce^{i\beta} & d \end{pmatrix}$ is given, where $a$, $b$, $c$, $d$, $\alpha$, and $\beta$ are real. The inverse of the matrix is:
\begin{multicols}{4}
\begin{enumerate}
     \item $\begin{pmatrix} ae^{ix} & -ce^{i\beta} \\ b & d \end{pmatrix}$
    \item $\begin{pmatrix} ae^{ix} & ce^{i\beta} \\ b & d \end{pmatrix}$
    \item $\begin{pmatrix} ae^{-ia} & b \\ ce^{-ip} & d \end{pmatrix}$
    \item $\begin{pmatrix} ae^{-ix} & ce^{-i\beta} \\ b & d \end{pmatrix}$
\end{enumerate}
\end{multicols}


% Question 3
\item The curl of a vector field $\overrightarrow{F}$ is $2\hat{x}$. Identify the appropriate vector field $\overrightarrow{F}$ from the choices given below:

\begin{multicols}{4}
    \begin{enumerate}
     \item $\overrightarrow{F} = 2z\hat{x} + 3z\hat{y} + 5y\hat{z}$
    \item $\overrightarrow{F} = 3z\hat{y} + 5y\hat{z}$
    \item $\overrightarrow{F} = 3x\hat{y} + 5y\hat{z}$
    \item $\overrightarrow{F} = 2\hat{x} + 5y\hat{z}$
    \end{enumerate}
\end{multicols}


% Question 4
\item A rigid body is rotating about its center of mass, fixed at the origin, with an angular velocity $\overrightarrow{\omega}$ and angular acceleration $\overrightarrow{\alpha}$. If the torque acting on it is $\overrightarrow{\tau}$ and its angular momentum is $\overrightarrow{L}$, the rate of change of its kinetic energy is:
\begin{multicols}{4}
\begin{enumerate}
  \item $\frac{1}{2}\overrightarrow{r}\cdot\overrightarrow{\omega}$
  \item $\frac{1}{2}\overrightarrow{L}\cdot\overrightarrow{\omega}$
    \item $\frac{1}{2}\brak{\overrightarrow{r}\cdot\overrightarrow{\omega} + \overrightarrow{L}\cdot\overrightarrow{\alpha}}$
    \item $\frac{1}{2}\overrightarrow{L}\cdot\overrightarrow{\alpha}$
\end{enumerate}
\end{multicols}


% Question 5
\item A cylinder of mass $M$ and radius $R$ is rolling down without slipping on an inclined plane of angle of inclination $\theta$. The number of generalized coordinates required to describe the motion of this system is:
\begin{multicols}{4}
\begin{enumerate}
     \item 1
    \item 2
    \item 4
    \item 6
\end{enumerate}
\end{multicols}


% Question 6
\item A parallel plate capacitor is being discharged. What is the direction of the energy flow in terms of the Poynting vector in the space between the plates?
\begin{center}
    \input{figs/fig.tex}
\end{center}
\begin{multicols}{2}
\begin{enumerate}
    \item Along the wire in the positive z-axis.
    \item Radially inward $\brak{-\overrightarrow{r}}$.
    \item Radially outward $\brak{\overrightarrow{r}}$.
    \item Circumferential direction $\brak{\phi}$.
\end{enumerate}
\end{multicols}



% Question 7
\item Unpolarized light falls from air to a planar air-glass interface (refractive index of glass is 1.5) and the reflected light is observed to be plane polarized. The polarization vector and the angle of incidence are:

\begin{enumerate}
        \item perpendicular to the plane of incidence and $\theta_{i}=42\degree$
        \item parallel to the plane of incidence and $\theta_{i}=56\degree$
        \item perpendicular to the plane of incidence and $\theta_{i}=56\degree$
        \item parallel to the plane of incidence and $\theta_{i}=42\degree$
\end{enumerate}


% Question 8
\item A finite wave train, of an unspecified nature, propagates along the positive $x$ axis with a constant speed $v$ and without any change of shape. The differential equation among the four listed below, whose solution it must be, is: 
   
\begin{multicols}{2}
\begin{enumerate}
       \item $\brak{\frac{\partial^2}{\partial x^2} - \frac{1}{v^2} \frac{\partial^2}{\partial t^2}} \psi\brak{x,t} = 0$
        \item $\brak{ \nabla^2 - \frac{1}{v^2} \frac{\partial^2}{\partial t^2} } \psi'\brak{\overrightarrow{r},t} = 0$
        \item $\brak{\nabla^z + a \frac{\partial}{\partial t}} \psi\brak{\vec{r},t} = 0$
        \item $\brak{-\frac{\hbar^2}{2m} \frac{\partial^2}{\partial x^2} - i \hbar \frac{\partial}{\partial t}} \psi\brak{\overrightarrow{r},t} = 0$

\end{enumerate}
\end{multicols}


% Question 9
\item Let $|\psi_0\rangle$ denote the ground state of the hydrogen atom. Choose the correct statement from those given below:
 
\begin{multicols}{2}
\begin{enumerate}
        \item $[L_x, L_y]|\psi_0\rangle = 0$
        \item $J^2 |\psi_0\rangle = 0$
        \item $\overrightarrow{L} \cdot \overrightarrow{S} |\psi_0\rangle \neq 0$
        \item $[S_x, S_y] |\psi_0\rangle = 0$
\end{enumerate}
\end{multicols}

% Question 10
\item Thermodynamic variables of a system can be volume $V$, pressure $P$, temperature $T$, number of particles $N$, internal energy $E$, and chemical potential $\mu$. For a system to be specified by Microcanonical (MC), Canonical (CE), and Grand Canonical (GC) ensembles, the parameters required for the respective ensembles are:
\begin{multicols}{2}
\begin{enumerate}
    \item MC: $\brak{N, V, T}$; CE: $\brak{E, V, N}$; GC: $\brak{V, T, \mu}$
    \item MC: $\brak{E, V, N}$; CE: $\brak{N, V, T}$; GC: $\brak{V, T, \mu}$
    \item MC: $\brak{V, T, \mu}$; CE: $\brak{N, V, T}$; GC: $\brak{E, V, N}$
    \item MC: $\brak{E, V, N}$; CE: $\brak{V, T, \mu}$; GC: $\brak{N, V, T}$
\end{enumerate}
\end{multicols}


% Question 11
\item The pressure versus temperature diagram of a given system at certain low temperature range is found to be parallel to the temperature axis in the liquid-to-solid transition region. The change in the specific volume remains constant in this region. The conclusion one can get from the above is
\begin{enumerate}
    \item The entropy of solid is zero in this temperature region.
    \item The entropy increases when the system goes from liquid to solid phase in this temperature region.
    \item  the entropy decreases when the system transforms from liquid to solid phase in this region of temperature.
    \item the change in entropy is zero in the liquid-to-solid transition region
\end{enumerate}

% Question 12
\item The radial wave function of the electrons in the state of $n=1$ and $l=0$ in a hydrogen atom is $R_{10}=\frac{2}{{a_{0}}^{\frac{3}{2}}}\exp\brak{-\frac{r}{a_{0}}}$ where $a_{0}$ is the Bohr radius. The most probable value of $r$ for an electron is
\begin{multicols}{4}
\begin{enumerate}
    \item $a_0$
    \item $2a_0$
    \item $4a_0$
    \item $8a_0$
\end{enumerate}
\end{multicols}




% Question 13
\item  The last two terms of the electronic configuration of manganese (Mn) atom is $3d^{5}4s^{2}$. The term factor of Mn$^{4+}$ is
\begin{multicols}{4}
\begin{enumerate}
    \item $^{4}D_{\frac{1}{2}}$
    \item $^{4}F_{\frac{3}{2}}$
    \item $^{3}F_{\frac{9}{2}}$
    \item $^{4}D_{\frac{7}{2}}$
\end{enumerate}
\end{multicols}



% Question 14
\item The coherence length of laser light is
    
\begin{enumerate}
        \item directly proportional to the length of the active lasing medium.
        \item directly proportional to the width of the spectral line.
        \item inversely proportional to the width of the spectral line.
        \item inversely proportional to the length of the active lasing medium.
\end{enumerate}


% Question 15
\item Metallic monovalent sodium crystallizes in a body-centered cubic (BCC) structure. If the length of the unit cell is $4\times10^{-8}$ cm, the concentration of conduction electrons in metallic sodium is
\begin{multicols}{4}
\begin{enumerate}
    \item $6.022\times10^{22}$ cm$^{-3}$
        \item $3.125\times10^{22}$ cm$^{-3}$
        \item $2.562\times10^{22}$ cm$^{-3}$
        \item $1.250\times10^{22}$ cm$^{-3}$
\end{enumerate}
\end{multicols}

% Question 16
\item  The plot of inverse magnetic susceptibility $1 / \chi$ versus temperature $T$ of an antiferromagnetic sample corresponds to
\begin{multicols}{2}
\begin{enumerate}
    \item \input{figs/figure.tex}
    \item \input{figs/figure2.tex}
    \item \input{figs/figure1.tex}
    \item \input{figs/figure3.tex}
    \end{enumerate}
\end{multicols}
% Question 17
\item   According to the quark model, the $\mathrm{K}^{+}$meson is composed of the following quarks
\begin{multicols}{4}
\begin{enumerate}
    \item u u d
    \item u $\overline{C}$
    \item u $\overline{S}$
    \item s $\overline{u}$
\end{enumerate}
\end{multicols}

\end{enumerate}
\end{document}




