\documentclass[journal,9pt,onecolumn]{IEEEtran}
\usepackage[a5paper, margin=8mm]{geometry}
%\usepackage{lmodern} % Ensure lmodern is loaded for pdflatex
\usepackage{tfrupee} % Include tfrupee package

\setlength{\headheight}{1cm} % Set the height of the header box
\setlength{\headsep}{0mm}     % Set the distance between the header box and the top of the text

\usepackage{gvv-book}
\usepackage{gvv}
\usepackage{cite}
\usepackage{amsmath,amssymb,amsfonts,amsthm}
\usepackage{algorithmic}
\usepackage{graphicx}
\usepackage{textcomp}
\usepackage{xcolor}
\usepackage{txfonts}
\usepackage{listings}
\usepackage{enumitem}
\usepackage{mathtools}
\usepackage{gensymb}
\usepackage{comment}
\usepackage[breaklinks=true]{hyperref}
\usepackage{tkz-euclide} 
\usepackage{listings}
% \usepackage{gvv}                                        
\def\inputGnumericTable{}                                 
\usepackage[latin1]{inputenc}                                
\usepackage{color}                                            
\usepackage{array}                                            
\usepackage{longtable}                                       
\usepackage{calc}                                             
\usepackage{multirow}                                         
\usepackage{hhline}                                           
\usepackage{ifthen}                                           
\usepackage{lscape}
\begin{document}

\bibliographystyle{IEEEtran}
\vspace{3cm}
\title{2009-XE 37-48}
\author{AI24BTECH11002 - K. Akshay Teja}
\maketitle
 %\newpage
 \bigskip
{\let\newpage\relax\maketitle}

\renewcommand{\thefigure}{\theenumi}
\renewcommand{\thetable}{\theenumi}
\setlength{\intextsep}{10pt} % Space between text and floats

\numberwithin{equation}{enumi}
\numberwithin{figure}{enumi}
\renewcommand{\thetable}{\theenumi}

\begin{enumerate}


% Question 37
\item Equal size spherical balls when packed together will yield maximum theoretical packing of
\begin{multicols}{4}    
\begin{enumerate}
     \item ~52\%
     \item ~68\%
     \item ~74\%
     \item ~86\%
\end{enumerate}
\end{multicols}

% Question 38
\item Steel containing 0.8\% carbon cooled under equilibrium conditions from molten state to room
temperature is soft, because it consists of lamellae of
\begin{multicols}{2}
\begin{enumerate}
    \item Ferrite and cementite
    \item Ferrite and austenite
    \item Ferrite and bainite
    \item Ferrite and martensite
\end{enumerate}
\end{multicols}


% Question 39
\item  Line broadening in X-ray diffraction pattern occurs on account of
\begin{multicols}{2}
\begin{enumerate}
    \item Coarse crystallite size
    \item Residual stresses
    \item Multiplicity of phases
    \item Coring of crystallites
    \end{enumerate}
\end{multicols}


% Question 40
\item  Inter-granular corrosion of austenitic stainless steel is promoted by
\begin{multicols}{2}    
\begin{enumerate}
    \item Fine grained microstructure 
    \item Coarse grained microstructure
    \item Soaking steel at 700$\degree$C in air
    \item Quenching from 1000$\degree$C
\end{enumerate}
\end{multicols}


% Question 41
\item Ferrites are preferred materials for use in high frequency applications (GHz range) as opposed to other ferromagnetic materials because ferrites also have 
\begin{multicols}{2}
\begin{enumerate}
    \item High permeability
    \item High electrical resistivity
    \item High saturation magnetization
    \item Low coercivity
\end{enumerate}
\end{multicols}


% Question 42
\item  During indirect intra-band transition, electrons undergo
\begin{enumerate}
    \item Change in energy and momentum
    \item Change in momentum but no change in energy
    \item Change neither in energy nor in momentum
    \item Change in energy but no change in momentum
\end{enumerate}

% Question 43
\item    A material has a band gap of 2.4 eV. Which of the following wavelengths of light will it absorb?  
\begin{multicols}{4}    
\begin{enumerate}
    \item 700 nm
    \item 550 nm
    \item 650 nm
    \item 400 nm
\end{enumerate}
\end{multicols}


% Question 44
\item Thermal conductivity of a material at a temperature greater than Debye temperature
\begin{enumerate}
    \item is independent of temperature
    \item decreases inversely with temperature
    \item increases linearly with temperature
    \item increases exponentially with temperature
\end{enumerate}


% Question 45
\item Match the following classes of materials given in Column I with the electron spin alignments in atoms shown in Column II.
\begin{table}[h!]
 	\centering
 	\begin{tabular}{|c|c|}
	\hline
	Vector&Description\\
	\hline
	Vector A&  \(\hat{i} - 2 \hat{j} + 3 \hat{k}\)\\
	\hline
	Vector B& \(2\hat{i} +3 \hat{j} -4\hat{k}\)\\
	\hline
	Vector C& \(\hat{i} -3\hat{j} +\hat{k}\)\\
	\hline
\end{tabular}

 	\label{tab:Ph-2009}
 \end{table}
\begin{multicols}{4}
\begin{enumerate}
    \item  P-3, Q-1, R-4, S-5
    \item  P-4, Q-2, R-5, S-3
    \item  P-3, Q-1, R-5, S-2
    \item  P-3, Q-2, R-4, S-1
\end{enumerate}
\end{multicols}

% Question 46
\item Match the following experimental techniques given in Column I with applications given in Column II.  
\begin{table}[h!]
 	\centering
 	\begin{tabular}{clcl}
&\textbf{Column I} & & \textbf{Column II}  \\
P & Differential Scanning Calorimetry & 1. & Dislocation studies \\
Q & Atomic Absorption Spectroscopy & 2. & Surface Topography \\
R & Scanning Electron Microscopy & 3. & Electrical Conductivity \\
S & Transmission Electron Microscopy & 4. & Trace Element Analysis \\
  &                             & 5. & Phase Transformation \\
\end{tabular}
 	\label{tab:Ph-2009}
 \end{table}
\begin{multicols}{4}
\begin{enumerate}
    \item  P-5, Q-4, R-2, S-1
    \item  P-5, Q-1, R-3, S-2
    \item  P-2, Q-5, R-3, S-1
    \item  P-1, Q-5, R-4, S-2
\end{enumerate}
\end{multicols}


% Question 47
\item Match the following materials given in Column I with their applications given in Column II.
\begin{table}[h!]
 	\centering
 	\begin{tabular}{clcl}
&\textbf{Column I} & & \textbf{Column II}  \\
P & Nylon & 1. & Electrical switch housing \\
Q & Urea formaldehyde & 2. & Conducting polymers \\
R & Polyaniline & 3. & Heating Element \\
S & Alumina & 4. & Gears for toys \\
&&5.&Polishing material
\end{tabular}
 	\label{tab:Ph-2009}
\end{table}
\begin{multicols}{4}    
\begin{enumerate}
    \item P-2, Q-4, R-3, S-5
    \item P-4, Q-1, R-2, S-5
    \item P-3, Q-4, R-2, S-1
    \item P-4, Q-5, R-3, S-2
\end{enumerate}
\end{multicols}


% Question 48
\item Match the following materials given in Column I with their applications given in Column II.
\begin{table}[h!]
 	\centering
 	\begin{tabular}{clcl}
&\textbf{Column I} & & \textbf{Column II}  \\
P & Silicon carbide fibre & 1. & Fibre glass boat \\
Q & Polyester fibre & 2. & Heating element \\
R & Thoria doped tungsten & 3. & Magnetic material \\
S & Nichrome & 4. & Electric bulb filament \\
&&5.&Armour material
\end{tabular}

 	\label{tab:Ph-2009}
\end{table}
\begin{multicols}{4}
\begin{enumerate}
    \item P-5, Q-1, R-3, S-2
    \item P-1, Q-5, R-4, S-2
    \item P-5, Q-3, R-2, S-1
    \item P-5, Q-1, R-4, S-2
\end{enumerate}
\end{multicols}

\end{enumerate}
\end{document}

