\documentclass[journal,9pt,onecolumn]{IEEEtran}
\usepackage[a5paper, margin=8mm]{geometry}
%\usepackage{lmodern} % Ensure lmodern is loaded for pdflatex
\usepackage{tfrupee} % Include tfrupee package

\setlength{\headheight}{1cm} % Set the height of the header box
\setlength{\headsep}{0mm}     % Set the distance between the header box and the top of the text

\usepackage{gvv-book}
\usepackage{gvv}
\usepackage{cite}
\usepackage{amsmath,amssymb,amsfonts,amsthm}
\usepackage{algorithmic}
\usepackage{graphicx}
\usepackage{textcomp}
\usepackage{xcolor}
\usepackage{txfonts}
\usepackage{listings}
\usepackage{enumitem}
\usepackage{mathtools}
\usepackage{gensymb}
\usepackage{comment}
\usepackage[breaklinks=true]{hyperref}
\usepackage{tkz-euclide} 
\usepackage{listings}
% \usepackage{gvv}                                        
\def\inputGnumericTable{}                                 
\usepackage[latin1]{inputenc}                                
\usepackage{color}                                            
\usepackage{array}                                            
\usepackage{longtable}                                       
\usepackage{calc}                                             
\usepackage{multirow}                                         
\usepackage{hhline}                                           
\usepackage{ifthen}                                           
\usepackage{lscape}
\begin{document}

\bibliographystyle{IEEEtran}
\vspace{3cm}
\title{2017-CE-1-13}
\author{AI24BTECH11002 - K. Akshay Teja}
\maketitle
 %\newpage
 \bigskip
{\let\newpage\relax\maketitle}

\renewcommand{\thefigure}{\theenumi}
\renewcommand{\thetable}{\theenumi}
\setlength{\intextsep}{10pt} % Space between text and floats

\numberwithin{equation}{enumi}
\numberwithin{figure}{enumi}
\renewcommand{\thetable}{\theenumi}


\begin{enumerate}

% Question 1
\item The matrix P is the inverse of a matrix Q. If I denotes the identity matrix, which one of the following options is correct?
\begin{multicols}{4}
    \begin{enumerate}
        \item PQ = I but QP $\ne$ I
        \item QP = I but PQ $\ne$ I
        \item PQ = I and QP = I
        \item PQ - QP = I
    \end{enumerate}
\end{multicols}

% Question 2
\item The number of parameters in the univariate exponential and Gaussian distributions, respectively, are:
\begin{multicols}{4}
    \begin{enumerate}
        \item $2$ and $2$
        \item $1$ and $2$
        \item $2$ and $1$
        \item $1$ and $1$
    \end{enumerate}
\end{multicols}


% Question 3
\item Let $x$ be a continuous variable defined over the interval $\brak{-\infty, \infty}$ and $f\brak{x} = e^{-x - e^{-x}}$. The integral $g\brak{x} = \int f\brak{x} \, dx$ is equal to:
    \begin{multicols}{4}
        \begin{enumerate}
            \item $e^{e^{-x}}$
            \item $e^{-e^{-x}}$
            \item $e^{-e^{x}}$
            \item $e^{-x}$
        \end{enumerate}
    \end{multicols}
    
% Question 4
\item An elastic bar of length $Z$, uniform cross-sectional area $A$, coefficient of thermal expansion $\alpha$, and Young's modulus $E$ is fixed at the two ends. The temperature of the bar is increased by $T$, resulting in an axial stress $\sigma$. Keeping all other parameters unchanged, if the length of the bar is doubled, the axial stress would be:
\begin{multicols}{4}
    \begin{enumerate}
        \item $\sigma$
        \item $2\sigma$
        \item $0.5 \sigma$
        \item $0.25 \alpha \sigma$
    \end{enumerate}
\end{multicols}

% Question 5
\item Statement: A simply supported beam is subjected to a uniformly distributed load. Which one of the following statements is true?
\begin{enumerate}
    \item Maximum or minimum shear force occurs where the curvature is zero.
    \item Maximum or minimum bending moment occurs where the shear force is zero.
    \item Maximum or minimum bending moment occurs where the curvature is zero.
    \item Maximum bending moment and maximum shear force occur at the same section.
\end{enumerate}

% Question 6
\item According to IS 456-2000, which one of the following statements about the depth of neutral axis $x_{u,bal}$ for a balanced reinforced concrete section is correct?
    \begin{enumerate}
        \item $x_{u,bal}$ depends on the grade of concrete only.
        \item $x_{u,bal}$ depends on the grade of steel only.
        \item $x_{u,bal}$ depends on both the grade of concrete and grade of steel.
        \item $x_{u,bal}$ does not depend on the grade of concrete and grade of steel.
    \end{enumerate}



% Question 7
\item The figure shows a two-hinged parabolic arch of span $L$ subjected to a uniformly distributed load of intensity $q$ per unit length. 
\begin{center}
    \resizebox{0.3\textwidth}{!}{%
\begin{circuitikz}
\tikzstyle{every node}=[font=\LARGE]



\draw [short] (7.5,12.5) -- (12.75,12.5);
\draw  (10.25,14.5) circle (2cm);
\draw [->, >=Stealth] (10.25,14.5) -- (8.5,13.5);
\draw [short] (8.5,13.5) -- (7.75,13.5);
\node [font=\LARGE] at (10.25,15) {P};
\node [font=\LARGE] at (10.25,17.25) {Q};
\node [font=\LARGE] at (7.75,13.75) {r};
\node at (10.25,14.5) [circ] {};
\node at (10.25,16.5) [circ] {};
\end{circuitikz}
}%


\end{center}
The maximum bending moment in the arch is ewual to
\begin{multicols}{4}
    \begin{enumerate}
        \item $\frac{qL^2}{8}$
        \item $\frac{qL^2}{12}$
        \item zero
        \item $\frac{qL^2}{10}$
    \end{enumerate}
\end{multicols}

% Question 8
 \item Group I lists the type of gain or loss of strength in soils. Group II lists the property or process responsible for the loss or gain of strength in soils.
\begin{table}[h!]
    \centering
    \begin{tabular}{|c|c|}
	\hline
	Vector&Description\\
	\hline
	Vector A&  \(\hat{i} - 2 \hat{j} + 3 \hat{k}\)\\
	\hline
	Vector B& \(2\hat{i} +3 \hat{j} -4\hat{k}\)\\
	\hline
	Vector C& \(\hat{i} -3\hat{j} +\hat{k}\)\\
	\hline
\end{tabular}

    \label{tab:CE-2017}
\end{table}\\
The correct match between group I and group II is
\begin{multicols}{2}
 \begin{enumerate}
        \item P-4, Q-1, R-2, S-3
        \item P-3, Q-1, R-2, S-4
        \item P-3, Q-2, R-1, S-4
        \item P-4, Q-2, R-1, S-3
    \end{enumerate}
\end{multicols}

    
% Question 9
\item A soil sample is subjected to a hydrostatic pressure, $\sigma$. The Mohr circle for any point in the soil sample would be:
    \begin{enumerate}
        \item a circle of radius $\sigma$ and center at the origin
        \item a circle of radius $\sigma$ and center at a distance $\sigma$ from the origin
        \item a point at a distance $\sigma$ from the origin
        \item a circle of diameter $\sigma$ and center at the origin
    \end{enumerate}

% Question 10
\item A strip footing is resting on the ground surface of a pure clay bed having an undrained cohesion $C_u$. The ultimate bearing capacity of the footing is equal to:
\begin{multicols}{4}
\begin{enumerate}
    \item $2\pi C_u$
    \item $\pi C_u$
    \item $\brak{\pi + 1} C_u$
    \item $\brak{\pi + 2} C_u$
\end{enumerate}
\end{multicols}


% Question 11
\item A uniformly distributed line load of $500$ kN/m is acting on the ground surface. Based on Boussinesq's theory, the ratio of vertical stress at a depth $2$ m to that at $4$ m, right below the line of loading, is:
\begin{multicols}{4}
\begin{enumerate}
    \item $0.25$
    \item $0.5$
    \item $2.0$
    \item $4.0$
\end{enumerate}
\end{multicols}

% Question 12
\item For a steady incompressible laminar flow between two infinite parallel stationary plates, the shear stress variation is:
\begin{multicols}{2}
\begin{enumerate}
    \item linear with zero value at the plates
    \item linear with zero value at the center
    \item quadratic with zero value at the plates
    \item quadratic with zero value at the center
\end{enumerate}
\end{multicols}


% Question 13
\item Statement: The reaction rate involving reactants $A$ and $B$ is given by $-k[A]^{\alpha}[B]^{\beta}$. Which one of the following statements is valid for the reaction to be a first-order reaction?
\begin{multicols}{4}
\begin{enumerate}
    \item $\alpha = 0$ and $\beta = 0$
    \item $\alpha = 1$ and $\beta = 0$
    \item $\alpha = 1$ and $\beta = 1$
    \item $\alpha = 1$ and $\beta = 2$
\end{enumerate}
\end{multicols}
\end{enumerate}
\end{document}

