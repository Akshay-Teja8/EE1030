\documentclass[journal,9pt,onecolumn]{IEEEtran}
\usepackage[a5paper, margin=8mm]{geometry}
%\usepackage{lmodern} % Ensure lmodern is loaded for pdflatex
\usepackage{tfrupee} % Include tfrupee package

\setlength{\headheight}{1cm} % Set the height of the header box
\setlength{\headsep}{0mm}     % Set the distance between the header box and the top of the text

\usepackage{gvv-book}
\usepackage{gvv}
\usepackage{cite}
\usepackage{amsmath,amssymb,amsfonts,amsthm}
\usepackage{algorithmic}
\usepackage{graphicx}
\usepackage{textcomp}
\usepackage{xcolor}
\usepackage{txfonts}
\usepackage{listings}
\usepackage{enumitem}
\usepackage{mathtools}
\usepackage{gensymb}
\usepackage{comment}
\usepackage[breaklinks=true]{hyperref}
\usepackage{tkz-euclide} 
\usepackage{listings}
% \usepackage{gvv}                                        
\def\inputGnumericTable{}                                 
\usepackage[latin1]{inputenc}                                
\usepackage{color}                                            
\usepackage{array}                                            
\usepackage{longtable}                                       
\usepackage{calc}                                             
\usepackage{multirow}                                         
\usepackage{hhline}                                           
\usepackage{ifthen}                                           
\usepackage{lscape}
\begin{document}

\bibliographystyle{IEEEtran}
\vspace{3cm}
\title{2009-ME-27-39}
\author{AI24BTECH11002 - K. Akshay Teja}
\maketitle
 %\newpage
 \bigskip
{\let\newpage\relax\maketitle}

\renewcommand{\thefigure}{\theenumi}
\renewcommand{\thetable}{\theenumi}
\setlength{\intextsep}{10pt} % Space between text and floats

\numberwithin{equation}{enumi}
\numberwithin{figure}{enumi}
\renewcommand{\thetable}{\theenumi}

\begin{enumerate}

\setcounter{enumi}{26}
% Question 27
\item One of the eigenvectors of the matrix $A = \begin{bmatrix} 2 & 2 \\ 1 & 3 \end{bmatrix}$ is:

\begin{multicols}{4}    
\begin{enumerate}
    \item $ \left\{\begin{array}{c}2 \\ -1 \end{array}\right\}$
    \item $ \left\{\begin{array}{c}2 \\ 1 \end{array}\right\}$
    \item $ \left\{\begin{array}{c}4 \\ 1 \end{array}\right\}$
    \item $ \left\{\begin{array}{c}1 \\ -1 \end{array}\right\}$
\end{enumerate}
\end{multicols}

% Question 28
\item The velocity vector of a flow field is given as $ \vec{V} = 2xy \, \hat{i} - x^2 \, \hat{j} $. The vorticity vector at $(1, 1, 1)$ is:
\begin{multicols}{4}
\begin{enumerate}
    \item $ 4 \, \hat{i} - \hat{j} $
    \item $ 4 \, \hat{i} - \hat{k} $
    \item $ \hat{i} - 4 \, \hat{j} $
    \item $ \hat{i} - 4 \, \hat{k} $
\end{enumerate}
\end{multicols}


% Question 29
\item  The Laplace transform of a function $f(t)$ is $\frac{1}{s^2(s+1)}$. The function $f(t)$ is:
\begin{multicols}{4}    
\begin{enumerate}
    \item $ t-1 + e^{-t} $
    \item $ t +1+ e^{-t} $
    \item $ -1 + e^{-t} $
    \item $ 2t + e^{t} $
\end{enumerate}
\end{multicols}


% Question 30
\item A box contains 2 washers, 3 nuts, and 4 bolts. Items are drawn from the box at random one at a time without replacement. The probability of drawing 2 washers first, followed by 3 nuts, and subsequently the 4 bolts is:
\begin{multicols}{4}    
\begin{enumerate}
    \item $ \frac{2}{315} $
    \item $ \frac{1}{630} $
    \item $ \frac{1}{1260} $
    \item $ \frac{1}{2520} $
\end{enumerate}
\end{multicols}


% Question 31
\item A band brake having a band width of 80 mm, drum diameter of 250 mm, coefficient of friction of 0.25, and angle of wrap of 270 degrees is required to exert a friction torque of 1000 Nm. The maximum tension (in kN) developed in the band is:
\begin{multicols}{4}
\begin{enumerate}
    \item $ 1.88 $
    \item $ 3.56 $
    \item $ 6.12 $
    \item $ 11.56 $
\end{enumerate}
\end{multicols}


% Question 32
\item A bracket (shown in the figure) is rigidly mounted on a wall using four rivets. Each rivet is 6 mm in diameter and has an effective length of 12 mm.
\begin{center}
    \resizebox{0.3\textwidth}{!}{%
\begin{circuitikz}
\tikzstyle{every node}=[font=\LARGE]



\draw [short] (7.5,12.5) -- (12.75,12.5);
\draw  (10.25,14.5) circle (2cm);
\draw [->, >=Stealth] (10.25,14.5) -- (8.5,13.5);
\draw [short] (8.5,13.5) -- (7.75,13.5);
\node [font=\LARGE] at (10.25,15) {P};
\node [font=\LARGE] at (10.25,17.25) {Q};
\node [font=\LARGE] at (7.75,13.75) {r};
\node at (10.25,14.5) [circ] {};
\node at (10.25,16.5) [circ] {};
\end{circuitikz}
}%


\end{center}
Direct stresss (in MPa) in the most heavily loaded rivet is
\begin{multicols}{4}    
\begin{enumerate}
    \item 4.4
    \item 8.8
    \item 17.6
    \item 35.2
\end{enumerate}
\end{multicols}

% Question 33
\item A mass $m$ attached to a spring is subjected to a harmonic force as shown in the figure. The amplitude of the forced motion is observed to be 50 mm. The value of $m$ (in kg) is:
\begin{center}
    \resizebox{0.3\textwidth}{!}{%
\begin{circuitikz}
\tikzstyle{every node}=[font=\LARGE]
\draw [short] (7.75,15) -- (7.75,11);
\draw [short] (7.75,11) -- (9.75,11);
\draw [short] (10.75,11) -- (12.75,11);
\draw [short] (12.75,11) -- (12.75,15);
\draw [short] (9.75,11) -- (9.75,12);
\draw [short] (9.75,12) -- (10.75,12);
\draw [short] (10.75,12) -- (10.75,11);
\draw [->, >=Stealth] (5.5,13.5) -- (5.5,15);
\draw [->, >=Stealth] (5.5,13.5) -- (7,13.5);
\node [font=\LARGE] at (6.25,15) {$V\brak{x}$};
\node [font=\LARGE] at (7,13) {x};
\node [font=\LARGE] at (10.25,12.5) {a};
\node [font=\LARGE] at (10.25,10.5) {L};
\node [font=\LARGE] at (11.5,11.5) {$V_0$};
\draw [<->, >=Stealth] (11,12) -- (11,11);
\draw [->, >=Stealth] (10.75,10.5) -- (12.75,10.5);
\draw [->, >=Stealth] (9.75,10.5) -- (7.75,10.5);
\end{circuitikz}
}%


\end{center}
\begin{multicols}{4}    
\begin{enumerate}
    \item $ 0.1 $
    \item $ 1.0 $
    \item $ 0.3 $
    \item $ 0.5 $
\end{enumerate}
\end{multicols}


% Question 34
\item For the epicyclic gear arrangement shown in the figure, $\omega_2 = 100$ rad/s clockwise (CW) and $\omega_{\Delta rms} = 80$ rad/s counterclockwise (CCW). The angular velocity $\omega_{\text{arm}}$ (in rad/s) is
\begin{center}
    \resizebox{0.3\textwidth}{!}{%
\begin{circuitikz}
\tikzstyle{every node}=[font=\LARGE]
\draw [line width=0.6pt, short] (7.5,17.25) -- (7.5,12.25);
\draw [line width=0.6pt, short] (7.5,12.25) -- (15.75,12.25);
\draw [line width=0.6pt, short] (7.5,14.75) -- (15.5,14.75);
\draw [line width=0.6pt, short] (7.5,14.75) .. controls (8.25,16.25) and (8.25,15.5) .. (8.75,14.75);
\node [font=\LARGE] at (6.75,16.75) {V};
\node [font=\LARGE] at (6.75,14.75) {0};
\node [font=\LARGE] at (16,12) {t};
\draw [line width=0.6pt, short] (9.75,14.75) .. controls (10.5,16.25) and (10.5,15.5) .. (11,14.75);
\draw [line width=0.6pt, short] (12,14.75) .. controls (12.75,16.25) and (12.75,15.5) .. (13.25,14.75);
\draw [line width=0.6pt, short] (14.25,14.75) .. controls (15,16.25) and (15,15.5) .. (15.5,14.75);
\end{circuitikz}
}%


\end{center}
\begin{multicols}{4}    
\begin{enumerate}
    \item $0$
    \item $70$ CW
    \item $140$ CCW
    \item $140$ CW
\end{enumerate}
\end{multicols}


% Question 35
\item A lightly loaded full journal bearing has a journal diameter of 50 mm, bush bore of 50.05 mm, and bush length of 20 mm. If the rotational speed of the journal is 1200 rpm and the average viscosity of the liquid lubricant is 0.03 Pa s, the power loss (in W) will be:
\begin{multicols}{4}
\begin{enumerate}
    \item $37$
    \item $74$
    \item $118$
    \item $237$
\end{enumerate}
\end{multicols}

% Question 36
\item  For the configuration shown, the angular velocity of 
link AB is 10 rad/s counterclockwise. The magnitude of the relative sliding velocity (in m s$^{-1}$) of slider B with respect to rigid link CD is
\begin{center}
    \begin{figure}[!h]
\centering
\resizebox{0.4\textwidth}{!}{%
\begin{circuitikz}
\tikzstyle{every node}=[font=\LARGE]
\draw [ line width=0.2pt ] (10.5,15.5) circle (3.5cm);
\draw [ line width=0.2pt](6,20) to[short] (16,20);
\draw [line width=0.2pt, short] (6,20) .. controls (5.5,19.5) and (5.5,19.5) .. (5.5,18.75);
\draw [ line width=0.2pt ] (5,19.25) rectangle (6,18.25);
\draw [ line width=0.2pt ] (5.5,18.75) circle (0.5cm);
\draw [line width=0.2pt, short] (10.5,15.5) -- (11.75,18.75);
\draw [line width=0.2pt, short] (9.25,18.75) -- (10.5,15.5);
\draw [ line width=0.2pt](11.75,20) to[short] (11.75,18.75);
\draw [ line width=0.2pt](9.25,20) to[short] (9.25,18.75);
\draw [ line width=0.2pt](14,15.5) to[short] (4.5,15.5);
\node [font=\Huge] at (10.5,16.75) {$45\circ$};
\draw [ line width=0.2pt](14.75,18) to[short] (16.5,18);
\draw [line width=0.2pt, ->, >=Stealth] (14.75,18) -- (13.75,17);
\node [font=\Huge] at (15,18.5) {$300 mm$};
\node [font=\Huge] at (7.25,20.5) {$200 mm$};
\node [font=\Huge] at (13,20.5) {$400 mm$};
\node [font=\Huge] at (17,20.5) {$400 N$};
\node [font=\Huge, rotate around={90:(0,0)}] at (4.5,17) {$150 mm$};
\draw [line width=0.2pt, ->, >=Stealth] (4.5,18.25) -- (4.5,19);
\draw [line width=0.2pt, ->, >=Stealth] (4.5,16) -- (4.5,15.5);
\draw [line width=0.2pt, ->, >=Stealth] (8.25,20.5) -- (10.5,20.5);
\draw [line width=0.2pt, ->, >=Stealth] (6,20.5) -- (5.5,20.5);
\draw [ line width=0.2pt](10.5,20.75) to[short] (10.5,12);
\draw [line width=0.2pt, ->, >=Stealth] (11.75,20.5) -- (10.5,20.5);
\draw [line width=0.2pt, ->, >=Stealth] (14,20.5) -- (16,20.5);
\draw [line width=0.2pt, ->, >=Stealth] (16,20.5) -- (16,20);
\draw [line width=0.2pt, ->, >=Stealth] (11.75,15.5) .. controls (11.75,14.5) and (11.25,14) .. (10.5,14.25) ;
\end{circuitikz}
}

\label{fig:my_label}
\end{figure}
\end{center}
\begin{multicols}{4}
\begin{enumerate}
    \item $0$
    \item $0.86$
    \item $1.25$
    \item $250$
\end{enumerate}
\end{multicols}


% Question 37
\item A smooth pipe of diameter 200 mm carries water. The pressure in the pipe at section $S_1$ (elevation: 10 m) is 50 kPa. At section $S_2$ (elevation: 12 m), the pressure is 20 kPa and the velocity is 2 m/s. The density of water is 1000 kg/m$^3$ and the acceleration due to gravity is 9.8 m/s$^2$. Which of the following is TRUE:
\begin{enumerate}
    \item flow is from $S_1$ to $S_2$ and head loss is 0.53 m
    \item flow is from $S_2$ to $S_1$ and head loss is 0.53 m
    \item flow is from $S_1$ to $S_2$ and head loss is 1.06 m
    \item flow is from $S_2$ to $S_1$ and head loss is 1.06 m
\end{enumerate}


% Question 38
\item Match the following:
\begin{table}[h!]
 	\centering
 	\begin{tabular}{|c|c|}
	\hline
	Vector&Description\\
	\hline
	Vector A&  \(\hat{i} - 2 \hat{j} + 3 \hat{k}\)\\
	\hline
	Vector B& \(2\hat{i} +3 \hat{j} -4\hat{k}\)\\
	\hline
	Vector C& \(\hat{i} -3\hat{j} +\hat{k}\)\\
	\hline
\end{tabular}

 	\label{tab:ME-2010}
 \end{table}
 Which of the following matches is correct?
\begin{multicols}{2}
\begin{enumerate}
    \item P-U; Q-X; R-V; S-Z; T-W
    \item P-W; Q-X; R-Z; S-U; T-V
    \item P-Y; Q-W; R-Z; S-U; T-X
    \item P-Y; Q-W; R-Z; S-U; T-V
\end{enumerate}
\end{multicols}




 % Question 39
\item A monoatomic ideal gas ($\gamma = 1.67$, molecular weight = 40) is compressed adiabatically from 0.1 MPa, 300 K to 0.2 MPa. The universal gas constant is 8.314 kJkmol$^{-1}$ K$^{-1}$. The work of compression of the gas (in kJkg$^{-1}$) is:
\begin{multicols}{4}
\begin{enumerate}
    \item $29.7$
    \item $19.9$
    \item $13.3$
    \item $0$
\end{enumerate}
\end{multicols}
\end{enumerate}
\end{document}

