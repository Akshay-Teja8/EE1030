\documentclass[journal,9pt,onecolumn]{IEEEtran}
\usepackage[a5paper, margin=8mm]{geometry}
%\usepackage{lmodern} % Ensure lmodern is loaded for pdflatex
\usepackage{tfrupee} % Include tfrupee package

\setlength{\headheight}{1cm} % Set the height of the header box
\setlength{\headsep}{0mm}     % Set the distance between the header box and the top of the text

\usepackage{gvv-book}
\usepackage{gvv}
\usepackage{cite}
\usepackage{amsmath,amssymb,amsfonts,amsthm}
\usepackage{algorithmic}
\usepackage{graphicx}
\usepackage{textcomp}
\usepackage{xcolor}
\usepackage{txfonts}
\usepackage{listings}
\usepackage{enumitem}
\usepackage{mathtools}
\usepackage{gensymb}
\usepackage{comment}
\usepackage[breaklinks=true]{hyperref}
\usepackage{tkz-euclide} 
\usepackage{listings}
% \usepackage{gvv}                                        
\def\inputGnumericTable{}                                 
\usepackage[latin1]{inputenc}                                
\usepackage{color}                                            
\usepackage{array}                                            
\usepackage{longtable}                                       
\usepackage{calc}                                             
\usepackage{multirow}                                         
\usepackage{hhline}                                           
\usepackage{ifthen}                                           
\usepackage{lscape}
\begin{document}

\bibliographystyle{IEEEtran}
\vspace{3cm}
\title{2011-AE-53-65}
\author{AI24BTECH11002 - K. Akshay Teja}
\maketitle
 %\newpage
 \bigskip
{\let\newpage\relax\maketitle}

\renewcommand{\thefigure}{\theenumi}
\renewcommand{\thetable}{\theenumi}
\setlength{\intextsep}{10pt} % Space between text and floats

\numberwithin{equation}{enumi}
\numberwithin{figure}{enumi}
\renewcommand{\thetable}{\theenumi}


\noindent
\textbf{Linked Answer Questions}\\
A thin-walled (thickness $\ll$ radius), hollow shaft of length 1 m and mean radius, $R=5 cm$ has to be designed such that it can transmit a torque, T=7 kN-m. A survey of different commercially available materials was made and following data was obtained from the suppliers ( $E$ : Young's modulus, $\tau_y$ : yield stress in shear, $\rho$; density):\\
\begin{table}[h!]
    \centering
    \begin{tabular}{|c|c|}
	\hline
	Variable & Description\\
	\hline
	Point P & \myvec{-2\\3\\5}\\
	\hline
	Point Q & \myvec{1\\2\\3}\\
	\hline
	Point R & \myvec{7\\0\\-1}\\
	\hline
\end{tabular}

    \label{AE-2011}
\end{table}
\begin{enumerate}
% Question 53
\item If you assume a factor of safety of 2, what should be the approximate thickness of such a shaft?
   \begin{multicols}{4}
   \begin{enumerate}
       \item 0.5 mm
       \item 1 mm
       \item 2 mm
       \item 4 mm
   \end{enumerate}
   \end{multicols}
\end{enumerate} 
\noindent
Prandtl's lifting line equation for a general wing is given by \\
$\alpha\brak{y_0}=\frac{\Gamma\brak{y_0}}{\pi U_{\infty} c\brak{y_0}}+\alpha_{L=0}\brak{y_0}+\frac{1}{4 \pi U_{\infty}} \int_{-\frac{b}{2}}^{\frac{b}{2}} \frac{\brak{\partial \Gamma / \partial y}}{y_0-y} d y$, 
where $U_{\infty}$ is the free-stream velocity, $\alpha$ is the angle of attack, $y_0$ is the spanwise location, $\alpha_{L-10}\brak{y_0}$ gives the spanwise variation of zero-lift angle, $c$ is the chord, $b$ is the span, and $\Gamma\brak{y_0}$ gives the spanwise variation of circulation.

\begin{enumerate}
\setcounter{enumi}{1}
% Question 54
\item The rate of change of circulation with angle of attack $\Gamma_\alpha=\frac{\partial \Gamma}{\partial \alpha}$ is:
   \begin{multicols}{2}
   \begin{enumerate}
       \item inversely proportional to $\alpha$
       \item independent of $\alpha$
       \item a linear function of $\alpha$
       \item a quadratic function of $\alpha$
   \end{enumerate}
   \end{multicols}

% Question 55
\item Given that $C_L \propto \int_{-\frac{b}{2}}^{\frac{b}{2}} \Gamma\brak{y} d y$, the corresponding lift curve-slope $\frac{\partial C_L}{\partial \alpha}$ is
   \begin{multicols}{2}
   \begin{enumerate}
       \item independent of $\alpha$
       \item a linear function of $\alpha$
       \item a quadratic function of $\alpha$
       \item a cubic function of $\alpha$
   \end{enumerate}
   \end{multicols}
\textbf{General Aptitude (GA) Questions}\\
% Question 56
\item Choose the word from the options given below that is most nearly opposite in meaning to the given word: Deference
   \begin{multicols}{4}
   \begin{enumerate}
       \item aversion
       \item resignation
       \item suspicion
       \item contempt
   \end{enumerate}
   \end{multicols}

% Question 57
\item Choose the most appropriate word(s) from the options given below to complete the following sentence: \\We lost confidence in him because he never \_\_\_\_\_\_ the grandiose promises he had made.
   \begin{multicols}{4}
   \begin{enumerate}
       \item delivered
       \item delivered on
       \item forgot
       \item reneged on
   \end{enumerate}
   \end{multicols}


% Question 58
\item Choose the word or phrase that best completes the sentence below:\\ \_\_\_\_\_\_ in the frozen wastes of Arctic takes special equipment.
   \begin{multicols}{4}
   \begin{enumerate}
       \item To survive
       \item Surviving
       \item Survival
       \item That survival
   \end{enumerate}
   \end{multicols}


% Question 59
\item In how many ways can 3 scholarships be awarded to 4 applicants, when each applicant can receive any number of scholarships?
   \begin{multicols}{4}
   \begin{enumerate}
       \item 4
       \item 12
       \item 64
       \item 81
   \end{enumerate}
   \end{multicols}

% Question 60
\item Choose the most appropriate word from the options given below to complete the following sentence:\\ The \_\_\_\_\_\_ of evidence was on the side of the plaintiff since all but one witness testified that his story was correct.
   \begin{multicols}{4}
   \begin{enumerate}
       \item paucity
       \item propensity
       \item preponderance
       \item accuracy
   \end{enumerate}
   \end{multicols}


% Question 61
\item If $\frac{2y+1}{y+2}<1$, then which of the following alternatives gives the CORRECT range of $y$?
   \begin{multicols}{4}
   \begin{enumerate}
       \item $-2 < y < 2$
       \item $-2 < y < 1$
       \item $-3 < y < 1$
       \item $-4 < y < 1$
   \end{enumerate}
   \end{multicols}

% Question 62
\item A student attempted to solve a quadratic equation in $x$ twice. However, in the first attempt, he incorrectly wrote the constant term and ended up with the roots as (4,3). In the second attempt, he incorrectly wrote down the coefficient of $x$ and got the roots as (3,2). Based on the above information, the roots of the correct quadratic equation are:
   \begin{multicols}{4}
   \begin{enumerate}
       \item \brak{-3, 4}
       \item \brak{3, -4}
       \item \brak{6, 1}
       \item \brak{4, 2}
   \end{enumerate}
   \end{multicols}

% Question 63
\item $L$, $M$, and $N$ are waiting in a queue meant for children to enter the zoo. There are 5 children between $L$ and $M$, and 8 children between $M$ and $N$. If there are 3 children ahead of $N$ and 21 children behind $L$, then what is the minimum number of children in the queue?
   \begin{multicols}{4}
   \begin{enumerate}
       \item 28
       \item 27
       \item 41
       \item 40
   \end{enumerate}
   \end{multicols}

% Question 64
\item Four archers $P$, $Q$, $R$, and $S$ try to hit a bull's eye during a tournament consisting of seven rounds. The final scores received by the players during the tournament are listed in the table below.
\newpage
\begin{figure}[!h]
\begin{center}
    \input{Fig/fig.tex}
\end{center}
\end{figure}

The final scores received by the players during the tournament are listed in the table below
   \begin{table}[!h]
       \centering
        \input{Tables/table1.tex}       
       \label{tab:AE-2011}
   \end{table}
   
   The most accurate and the most consistent players during the tournament are respectively:
   \begin{multicols}{4}
   \begin{enumerate}
       \item P and S
       \item Q and R
       \item Q and Q
       \item R and Q
   \end{enumerate}
   \end{multicols}

% Question 65

\item Nimbus clouds are dark and ragged, stratus clouds appear dull in colour a nd cover the entire
sky. Cirrus clouds are thin and delicate, whereas cumulus clouds look like cotton balls.\\
	It can be inferred from the passage that:
\begin{enumerate}
    \item A cumulus cloud on the ground is called fog.
    \item It is easy to predict the weather by studying clouds.
    \item Clouds are generally of very different shapes, sizes, and mass.
    \item There are four basic cloud types: stratus, nimbus, cumulus, and cirrus.
\end{enumerate}
\end{enumerate}
\end{document}
