\let\negmedspace\undefined
\let\negthickspace\undefined
\documentclass[journal,9pt,twocolumn]{IEEEtran}
\usepackage[a5paper, margin=8mm]{geometry}
%\usepackage{lmodern} % Ensure lmodern is loaded for pdflatex
\usepackage{tfrupee} % Include tfrupee package

\setlength{\headheight}{1cm} % Set the height of the header box
\setlength{\headsep}{0mm}     % Set the distance between the header box and the top of the text

\usepackage{gvv-book}
\usepackage{gvv}
\usepackage{cite}
\usepackage{amsmath,amssymb,amsfonts,amsthm}
\usepackage{algorithmic}
\usepackage{graphicx}
\usepackage{textcomp}
\usepackage{xcolor}
\usepackage{txfonts}
\usepackage{listings}
\usepackage{enumitem}
\usepackage{mathtools}
\usepackage{gensymb}
\usepackage{comment}
\usepackage[breaklinks=true]{hyperref}
\usepackage{tkz-euclide} 
\usepackage{listings}
% \usepackage{gvv}                                        
\def\inputGnumericTable{}                                 
\usepackage[latin1]{inputenc}                                
\usepackage{color}                                            
\usepackage{array}                                            
\usepackage{longtable}                                       
\usepackage{calc}                                             
\usepackage{multirow}                                         
\usepackage{hhline}                                           
\usepackage{ifthen}                                           
\usepackage{lscape}
\begin{document}

\bibliographystyle{IEEEtran}
\vspace{3cm}

\title{2020-Sep-3 Shift-2}
\author{AI24BTECH11002 - K. Akshay Teja}
% \maketitle
% \newpage
% \bigskip
{\let\newpage\relax\maketitle}

\renewcommand{\thefigure}{\theenumi}
\renewcommand{\thetable}{\theenumi}
\setlength{\intextsep}{10pt} % Space between text and floats

\numberwithin{equation}{enumi}
\numberwithin{figure}{enumi}
\renewcommand{\thetable}{\theenumi}

\begin{enumerate}
%1
    \item  If $x^3dy + xydx = x^2dy + 2ydx$; $y(2) = e$ and $x > 1$, then $y(4)$ is equal to:
        
        \begin{multicols}{4}
\begin{enumerate}
    \item $\frac{\sqrt{e}}{2}$
    \item $\frac{3}{2} \sqrt{e}$
    \item $\frac{1}{2} + \sqrt{e}$
    \item $\frac{3}{2} + \sqrt{e}$
\end{enumerate}
\end{multicols}

%2
\item Let $A$ be a $3 \times 3$ matrix such that adj $A = \myvec{2 &-1& 1\\-1& 0& 2\\1& -2& -1}$ 
and $B =$ adj(adj $ A$). If $\abs{A} = \lambda$ and $\abs{(B^{-1})^T} = \mu$, then the ordered pair, $(\abs{\lambda}, \mu)$ is equal to:

        \begin{multicols}{2}

\begin{enumerate}
    \item $\brak{9, \frac{1}{81}}$
    \item $\brak{9, \frac{1}{9}}$
    \item $\brak{3, \frac{1}{81}}$
    \item $\brak{3, 81}$
\end{enumerate}
\end{multicols}

%3
\item Let $a, b, c \in \mathbb{R}$ be such that $a^2 + b^2 + c^2 = 1$, if $a \cos \theta = b \cos\brak{\theta + \frac{2\pi}{3}} = c \cos\brak{\theta + \frac{4\pi}{3}}$, where $\theta = \frac{\pi}{9}$, then the angle between the vectors $a\hat{i} + b\hat{j} + c\hat{k}$ and $b\hat{i} + c\hat{j}+ a\hat{k} $ is:
        \begin{multicols}{4}

\begin{enumerate}
    \item $\frac{\pi}{2}$
    \item $\frac{2\pi}{3}$
    \item $\frac{\pi}{9}$
    \item $0$
\end{enumerate}
\end{multicols}

%4
\item Suppose $f(x)$ is a polynomial of degree four, having critical points at $(-1, 0, 1)$. If $T = \{ x \in \mathbb{R} \mid f(x) = f(0) \}$, then the sum of squares of all the elements of $T$ is:
        \begin{multicols}{4}

\begin{enumerate}
    \item 6
    \item 2
    \item 8
    \item 4
\end{enumerate}
\end{multicols}

%5
\item If the value of the integral $ \int_{0}^{\frac{1}{2}} \frac{x^2}{\brak{1 - x^2}^{\frac{3}{2}}}$ is $\frac{k}{6}$, then $k$ is equal to:
        
        \begin{multicols}{2}
\begin{enumerate}
    \item $2\sqrt{3} + \pi$
    \item $3\sqrt{2} + \pi$
    \item $3\sqrt{2} - \pi$
    \item $2\sqrt{3} - \pi$
\end{enumerate}
\end{multicols}

%6
 \item If the term independent of $x$ in the expansion of $\brak{ \brak{ \frac{3}{2} } x^2 - \frac{1}{3x} }^9$ is $k$, then $18k$ is equal to:
        \begin{multicols}{4}

\begin{enumerate}
    \item 5
    \item 9
    \item 7
    \item 11
\end{enumerate}
\end{multicols}

%7
\item If a triangle $ABC$ has vertices $A\brak{-1, 7}$, $B\brak{-7, 1}$, and $C\brak{5, -5}$, then its orthocentre has coordinates:
        \begin{multicols}{2}

\begin{enumerate}
    \item $\brak{-3, 3}$
    \item $\brak{-\frac{3}{5}, \frac{3}{5}}$
    \item $\brak{\frac{3}{5}, -\frac{3}{5}}$
    \item $\brak{3, -3}$
\end{enumerate}
\end{multicols}

%8
\item  Let $e_1$ and $e_2$ be the eccentricities of the ellipse, $\frac{x^2}{25} + \frac{y^2}{b^2} = 1$ (where $b < 5$) and the hyperbola, $\frac{x^2}{16} - \frac{y^2}{b^2} = 1$ respectively, satisfying $e_1 e_2 = 1$. If $\alpha$ and $\beta$ are the distances between the foci of the ellipse and the foci of the hyperbola respectively, then the ordered pair $\brak{\alpha, \beta}$ is equal to:
        \begin{multicols}{2}

\begin{enumerate}
    \item $\brak{8, 12}$
    \item $\brak{\frac{24}{5}, 10}$
    \item $\brak{\frac{20}{3}, 12}$
    \item $\brak{8, 10}$
\end{enumerate}
\end{multicols}

%9
\item If $z_1, z_2$ are complex numbers such that Re$(z_1) = |z_1 - 1|$, Re$(z_2) = \abs{z_2 - 1}$ and $\arg\brak{z_1 - z_2} = \frac{\pi}{6}$, then Im$(z_1 + z_2)$ is equal to:
        \begin{multicols}{4}

\begin{enumerate}
    \item $2\sqrt{3}$
    \item $\frac{2}{\sqrt{3}}$
    \item $\frac{1}{\sqrt{3}}$
    \item $\frac{\sqrt{3}}{2}$
\end{enumerate}
\end{multicols}

%10
\item The set of all real values of $\lambda$ for which the quadratic equations, $\brak{\lambda^2 + 1}x^2 - 4\lambda x + 2 = 0$ always have exactly one root in the interval $\brak{0, 1}$ is:
        
        \begin{multicols}{2}
\begin{enumerate}
    \item $\brak{-3, -1}$
    \item $\brak{2, 4}$
    \item $\brak{1, 3}$
    \item $\brak{0, 2}$
\end{enumerate}
\end{multicols}

%11
\item Let the latus rectum of the parabola $y^2 = 4x$ be the common chord to the circles $C_1$ and $C_2$, each of them having radius $2\sqrt{5}$. Then, the distance between the centres of the circles $C_1$ and $C_2$ is:

\begin{multicols}{4}
\begin{enumerate}
    \item $8$
    \item $8\sqrt{5}$
    \item $4\sqrt{5}$
    \item $12$
\end{enumerate}
\end{multicols}

%12
\item The plane which bisects the line joining the points $\brak{4, -2, 3}$ and $\brak{2, 4, -1}$ at right angles also passes through the point:

\begin{multicols}{2}
\begin{enumerate}
    \item $\brak{0, -1, 1}$
    \item $\brak{4, 0, 1}$
    \item $\brak{4, 0, -1}$
    \item $\brak{0, 1, -1}$
\end{enumerate}
\end{multicols}

%13
\item $\lim_{x \to a} \frac{\brak{a + 2x}^{\frac{1}{3}} - \brak{3x}^{\frac{1}{3}}}{\brak{3a + x}^{\frac{1}{3}} - \brak{4a}^{\frac{1}{3}}}$ is equal to:
\begin{multicols}{2}
\begin{enumerate}
    \item $\frac{2}{9} \brak{\frac{4}{3}}$
    \item $\frac{2}{3} \brak{\frac{4}{3}}$
    \item $\brak{\frac{2}{3}} \brak{\frac{2}{9}}^{\frac{1}{3}}$
    \item $\brak{\frac{2}{9}} \brak{\frac{2}{3}}^{\frac{1}{3}}$
\end{enumerate}
\end{multicols}

%14
\item Let $x_{i} \brak{1 \leq i \leq 10}$ be ten observations of a random variable $X$. If
$\sum_{i=1}^{10}\brak{x_{i}-p}=3$
and
$\sum_{i=1}^{10}\brak{x_{i}-p}^{2}=9$
where $0 \neq p \in \mathbb{R}$, then the standard deviation of these observations is:
\begin{multicols}{4}
\begin{enumerate}
    \item  $\frac{7}{10}$
    \item $\frac{9}{10}$
    \item $\sqrt{\frac{3}{5}}$
    \item  $\frac{4}{5}$
\end{enumerate}
\end{multicols}

%15
\item  The probability that a randomly chosen 5-digit number is made from exactly two digits is:
\begin{multicols}{4}
\begin{enumerate}
    \item $\frac{134}{10^4}$
    \item $\frac{121}{10^4}$
    \item $\frac{135}{10^4}$
    \item $\frac{50}{10^4}$
\end{enumerate}
\end{multicols}

\end{enumerate}
\end{document}

