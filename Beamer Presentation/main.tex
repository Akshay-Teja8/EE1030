\documentclass{beamer}
\mode<presentation>
\usepackage{amsmath}
\usepackage{amssymb}
%\usepackage{advdate}
\usepackage{adjustbox}
\usepackage{subcaption}
\usepackage{enumitem}
\usepackage{multicol}
\usepackage{mathtools}
\usepackage{listings}
\usepackage{url}
\usepackage{minted}
% \usepackage{gvv}

\usepackage{tcolorbox}
\tcbuselibrary{minted,breakable,xparse,skins}



\definecolor{bg}{gray}{0.95}
\DeclareTCBListing{mintedbox}{O{}m!O{}}{%
  breakable=true,
  listing engine=minted,
  listing only,
  minted language=#2,
  minted style=default,
  minted options={%
    linenos,
    gobble=0,
    breaklines=true,
    breakafter=,,
    fontsize=\scriptsize,
    numbersep=8pt,
    #1},
  boxsep=0pt,
  left skip=0pt,
  right skip=0pt,
  left=25pt,
  right=0pt,
  top=3pt,
  bottom=3pt,
  arc=5pt,
  leftrule=0pt,
  rightrule=0pt,
  bottomrule=2pt,
  toprule=2pt,
  colback=bg,
  colframe=orange!70,
  enhanced,
  overlay={%
    \begin{tcbclipinterior}
    \fill[orange!20!white] (frame.south west) rectangle ([xshift=20pt]frame.north west);
    \end{tcbclipinterior}},
  #3,
}


\def\UrlBreaks{\do\/\do-}
\usetheme{Madrid}
\usecolortheme{lily}
\setbeamertemplate{footline}
{
  \leavevmode%
  \hbox{%
  \begin{beamercolorbox}[wd=\paperwidth,ht=2.25ex,dp=1ex,right]{author in head/foot}%
    \insertframenumber{} / \inserttotalframenumber\hspace*{2ex} 
  \end{beamercolorbox}}%
  \vskip0pt%
}
\setbeamertemplate{navigation symbols}{}

\providecommand{\nCr}[2]{\,^{#1}C_{#2}} % nCr 
\providecommand{\nPr}[2]{\,^{#1}P_{#2}} % nPr
\providecommand{\mbf}{\mathbf}
\providecommand{\pr}[1]{\ensuremath{\Pr\left(#1\right)}}
\providecommand{\qfunc}[1]{\ensuremath{Q\left(#1\right)}}
\providecommand{\sbrak}[1]{\ensuremath{{}\left[#1\right]}}
\providecommand{\lsbrak}[1]{\ensuremath{{}\left[#1\right.}}
\providecommand{\rsbrak}[1]{\ensuremath{{}\left.#1\right]}}
\providecommand{\brak}[1]{\ensuremath{\left(#1\right)}}
\providecommand{\lbrak}[1]{\ensuremath{\left(#1\right.}}
\providecommand{\rbrak}[1]{\ensuremath{\left.#1\right)}}
\providecommand{\cbrak}[1]{\ensuremath{\left\{#1\right\}}}
\providecommand{\lcbrak}[1]{\ensuremath{\left\{#1\right.}}
\providecommand{\rcbrak}[1]{\ensuremath{\left.#1\right\}}}
\theoremstyle{remark}
\newtheorem{rem}{Remark}
\newcommand{\sgn}{\mathop{\mathrm{sgn}}}
\providecommand{\abs}[1]{\left\vert#1\right\vert}
\providecommand{\res}[1]{\Res\displaylimits_{#1}} 
\providecommand{\norm}[1]{\lVert#1\rVert}
\providecommand{\mtx}[1]{\mathbf{#1}}
\providecommand{\mean}[1]{E\left[ #1 \right]}
\providecommand{\fourier}{\overset{\mathcal{F}}{ \rightleftharpoons}}
%\providecommand{\hilbert}{\overset{\mathcal{H}}{ \rightleftharpoons}}
\providecommand{\system}{\overset{\mathcal{H}}{ \longleftrightarrow}}
	%\newcommand{\solution}[2]{\textbf{Solution:}{#1}}
%\newcommand{\solution}{\noindent \textbf{Solution: }}
\providecommand{\dec}[2]{\ensuremath{\overset{#1}{\underset{#2}{\gtrless}}}}
\newcommand{\myvec}[1]{\ensuremath{\begin{pmatrix}#1\end{pmatrix}}}
\let\vec\mathbf

\lstset{
%language=C,
frame=single, 
breaklines=true,
columns=fullflexible
}

\numberwithin{equation}{section}
% Title information
\title{Assignment on Collinearity of Points}
\author{K. Akshay Teja \\ AI24BTECH11002\\IIT Hyderabad}
\date{November 5, 2024}

\begin{document}


\begin{frame}
    \titlepage
\end{frame}


\begin{frame}{Outline}
    \tableofcontents
\end{frame}


\section{Problem}
\begin{frame}{Problem Statement}
    Show that the points 
    \begin{align}
    P = \begin{pmatrix} -2 \\ 3 \\ 5 \end{pmatrix}, \quad Q = \begin{pmatrix} 1 \\ 2 \\ 3 \end{pmatrix}, \quad R = \begin{pmatrix} 7 \\ 0 \\ -1 \end{pmatrix}
    \end{align}
    are collinear.
\end{frame}


\section{Solution}


\subsection{Variable Description}
\begin{frame}{Variable Description}
    Define the coordinates for each point as follows:
    \begin{align}
    P = \begin{pmatrix} -2 \\ 3 \\ 5 \end{pmatrix}, \quad Q = \begin{pmatrix} 1 \\ 2 \\ 3 \end{pmatrix}, \quad R = \begin{pmatrix} 7 \\ 0 \\ -1 \end{pmatrix}
    \end{align}
    \begin{table}[h!]
        \centering
	\begin{tabular}{|c|c|}
	\hline
	Variable & Description\\
	\hline
	Point P & \myvec{-2\\3\\5}\\
	\hline
	Point Q & \myvec{1\\2\\3}\\
	\hline
	Point R & \myvec{7\\0\\-1}\\
	\hline
\end{tabular}

        \caption{Coordinates of Points P, Q, and R}
    \end{table}
\end{frame}


\subsection{Collinearity Condition}
\begin{frame}{Collinearity Condition}
    Points $ P $, $ Q $, and $ R $ are collinear if:
	\begin{align}
    \text{rank} \begin{pmatrix} P & Q & R \end{pmatrix}^\top = 2
	\end{align}
\end{frame}


\subsection{Matrix and Row Reduction}
\begin{frame}{Matrix Formulation and Row Reduction}
    Write the coordinates as a matrix:
	\begin{align}
    	\begin{pmatrix}
    	-2 & 3 & 5 \\
	    1 & 2 & 3 \\
	    7 & 0 & -1
	\end{pmatrix}
	\end{align}
    Perform row operations:
	\begin{align}
	\xleftrightarrow[]{R_2 \leftarrow 2 R_2 + R_3} \begin{pmatrix} -2 & 3 & 5 \\ 0 & 7 & 11 \\ 7 & 0 & -1 \end{pmatrix}\\    
	\xleftrightarrow[]{R_3 \leftarrow 2 R_3 + 7 R_1} \begin{pmatrix} -2 & 3 & 5 \\ 0 & 7 & 11 \\ 0 & 21 & 33 \end{pmatrix}\\
    	\xleftrightarrow[]{R_3 \leftarrow R_3 - 3 R_2 } \begin{pmatrix} -2 & 3 & 5 \\ 0 & 7 & 11 \\ 0 & 0 & 0 \end{pmatrix}
	\end{align}
\end{frame}


\section{Conclusion}
\begin{frame}{Conclusion}
    Since the matrix has a rank of 2, the points $ P $, $ Q $, and $ R $ are collinear.
\end{frame}

\begin{frame}{Visualization}
    \begin{figure}
        \centering
        \includegraphics[width=0.8\textwidth]{fig/fig.png} 
	\caption{Plot of Points P, Q, and R}
    \end{figure}
\end{frame}


\section{Codes}
\subsection{Generating points on line using C}

\begin{frame}[fragile,allowframebreaks]
\frametitle{Generating points on line using C}
\begin{mintedbox}{c}[break at=.8\textheight]
#include<stdio.h>

int main(){
	double P[]={-2,3,5},Q[]={1,2,3},R[]={7,0,-1};
	
	printf("P %.2lf %.2lf %.2lf\n",P[0],P[1],P[2]);
	printf("Q %.2lf %.2lf %.2lf\n",Q[0],Q[1],Q[2]);
	printf("R %.2lf %.2lf %.2lf\n",R[0],R[1],R[2]);
	int numberOfValues=100;

	double x_values[numberOfValues],y_values[numberOfValues],z_values[numberOfValues];

	for(int i=0;i<numberOfValues;i++){
		double t=(double)i/numberOfValues;
		x_values[i]=P[0]+t*(R[0]-P[0]);
		y_values[i]=P[1]+t*(R[1]-P[1]);
		z_values[i]=P[2]+t*(R[2]-P[2]);
	}
	   for (int i = 0; i < numberOfValues; i++) {
        printf("%.2f %.2f %.2f\n", x_values[i], y_values[i], z_values[i]);
    }

    return 0;
}

  \end{mintedbox}
\end{frame}

 \subsection{Plotting the figure using Python}
\begin{frame}[fragile,allowframebreaks]
\frametitle{Plotting the figure using Python}
   \begin{mintedbox}{Python}[break at=.8\textheight]
import numpy as np
import matplotlib.pyplot as plt
from mpl_toolkits.mplot3d import Axes3D
import subprocess

result = subprocess.run(['./code'],stdout = subprocess.PIPE,text=True)
output = result.stdout.strip().split('\n')

P = np.fromstring(output[0].replace('P ',''),sep=' ')
Q = np.fromstring(output[1].replace('Q ',''),sep=' ')
R = np.fromstring(output[2].replace('R ',''),sep=' ')

store=np.genfromtxt(output[3:],delimiter='')
x_values,y_values,z_values = store.T


fig = plt.figure()
ax = fig.add_subplot(111, projection='3d')

ax.scatter(*P, color='r', label='P')
ax.scatter(*Q, color='g', label='Q')
ax.scatter(*R, color='b', label='R')

ax.text(P[0]+0.2,P[1],P[2],'P(-2,3,5)',color='black', ha='left')
ax.text(Q[0]+0.2,Q[1],Q[2],'Q(1,2,3)',color='black', ha='left')
ax.text(R[0]+0.2,R[1],R[2],'R(7,0,-1)',color='black', ha='left')

ax.plot(x_values,y_values,z_values,color='k',label='Line through P,Q,R')

ax.set_xlabel('X')
ax.set_ylabel('Y')
ax.set_zlabel('Z')

plt.title('Points P, Q and R')
plt.grid(True)
plt.savefig('/home/akshay-teja-kondi/gvv/Assignment3/fig/fig.png')
\end{mintedbox}
\end{frame}
\end{document}
